\documentclass[a4paper,10pt]{report}
\usepackage[utf8]{inputenc}
\usepackage{amsmath}
\usepackage{amssymb}
\usepackage{amsthm}
\usepackage{mathtools}
\usepackage{fancyhdr}
\usepackage{enumitem}
\usepackage[top=1in,left=1in,right=1in]{geometry}
\usepackage{mathrsfs}
\usepackage{bm}

\usepackage{bbm}
\usepackage{tikz-cd}
\usepackage{stackengine}
\usepackage{Math_Symbols} 
\usepackage{ifpdf}
\ifpdf
%\usepackage[pdftex]{graphicx}
%\else
%\usepackage[dvips]{graphicx}
%\fi

\setenumerate{listparindent=\parindent,topsep=\parskip}
% \setlist[enumerate]{topsep=\parskip}
\setlist[enumerate,2]{label=(\arabic*)}
\setlist[enumerate,3]{label=(\alph*)}

% \newcommand{\set}[1]{{\{#1\}}}
\newcommand{\ggen}[1]{\langle#1\rangle}
\newcommand{\pn}[2]{||#1||_{#2}}
\newcommand{\bpn}[2]{\left|\left|#1\right|\right|_{#2}}
\newcommand{\norm}[1]{||#1||}
\newcommand{\bnorm}[1]{\left|\left|#1\right|\right|}
\DeclarePairedDelimiter{\ceil}{\lceil}{\rceil}
\DeclarePairedDelimiter{\floor}{\lfloor}{\rfloor}
\DeclarePairedDelimiter{\set}{\{}{\}}
\DeclarePairedDelimiter{\abs}{|}{|}
\DeclarePairedDelimiter{\ket}{|}{\rangle}
\DeclarePairedDelimiter{\bra}{\langle}{|}

\newcommand{\ol}[1]{\overline{#1}}

\renewcommand{\mod}{\text{ mod }}

\renewcommand{\O}{\operatorname{O}} % Bound otherwise
\renewcommand{\o}{\operatorname{o}}
\newcommand{\T}{\text{yes}}
\newcommand{\F}{\text{no}}

\newcommand{\Z}{\mathbb{Z}}
\newcommand{\N}{\mathbb{N}}
\newcommand{\C}{\mathbb{C}}
\newcommand{\Q}{\mathbb{Q}}
\newcommand{\textns}{\otimes}
\newcommand{\rar}[2][]{\overset{#2}{\underset{#1}{\longrightarrow}}}

\DeclareMathOperator{\img}{img}
\DeclareMathOperator{\fop}{int}
\DeclareMathOperator{\fcl}{cl}
% \DeclareMathOperator{\lg}{lg}
\DeclareMathOperator{\vspan}{span}
\DeclareMathOperator{\rng}{rng}
\DeclareMathOperator{\Rng}{Rng}
\DeclareMathOperator{\Cov}{Cov}
\DeclareMathOperator{\Var}{Var}
\DeclareMathOperator{\Bernoulli}{Bernoulli}
\DeclareMathOperator{\Normal}{Normal}
\DeclareMathOperator{\Uniform}{Uniform}
\DeclareMathOperator{\Binom}{Binomial}
\DeclareMathOperator{\mgf}{mgf}
\DeclareMathOperator{\Supp}{Supp}

\newcommand{\cat}[1]{(\bm{#1})}

\providecommand{\Alpha}{A}

\newtheorem*{lemma*}{Lemma}

% Roman numerals
\makeatletter
\newcommand{\Romnum}[1]{\expandafter\@slowromancap\romannumeral #1@}
\makeatother
\newcommand{\factor}[1]{\text{\Romnum{#1}}}

\begin{document}
% \maketitle

\pagestyle{fancy}	
\fancyhf{} % Reset headers and footers
\lhead{NAMES GO HERE\\
Functional Analysis 2\\
\today}
\setlength{\headheight}{60pt}

\begin{center}
	\textbf{Homework 13}
\end{center}
\begin{enumerate}
		\setcounter{enumi}{108}
	\item
		\begin{enumerate}
			\item 
			\item 
			\item 
		\end{enumerate}
	\item	Suppose $\Gamma$ is a countable discrete group,
		and suppose $\varphi : L\Gamma \to L\Gamma$ is a normal completely positive map.
		Prove that $f : \Gamma \to \C$ given by $f(g) := \tr_{L\Gamma}(\varphi(\lambda_g)\lambda_g^*)$
		is a positive definite function.
		\begin{proof}
			Let $n \in \N$ and $g_1, \dots, g_n \in \Gamma$.
			Let $A \in M_n(\C)$ be the matrix with entries $a_{ij} = f(g_i^{-1}g_j)$.
			We want to show $A \ge 0$.
			Note that since $\tr$ is tracial, we can rewrite the entries as follows:
			\begin{align*}
				a_{ij} & = \tr_{L\Gamma}(\varphi(\lambda_{g_i^{-1}g_j})\lambda_{g_i^{-1}g_j}^*) \\
				 & = \tr_{L\Gamma}(\varphi(\lambda_{g_i}^* \lambda_{g_j})
				 \lambda_{g_j}^*\lambda_{g_i}) \\
				 & = \tr_{L\Gamma}(\lambda_{g_i}\varphi(\lambda_{g_i}^* \lambda_{g_j})
				 \lambda_{g_j}^*). \\
			\intertext{Let $x \in \C^n$. Then}
				\ggen{Ax, x} & = \sum_{i,j=1}^n{\overline{x}_i a_{ij} x_j} \\
				 & = \sum_{i,j=1}^n{\overline{x}_i \tr_{L\Gamma}(\lambda_{g_i}
				 \varphi(\lambda_{g_i}^* \lambda_{g_j})\lambda_{g_j}^*) x_j} \\
				 & = \tr_{L\Gamma}\left( \sum_{i,j=1}^n{\overline{x}_i \lambda_{g_i}
				 \varphi(\lambda_{g_i}^* \lambda_{g_j})\lambda_{g_j}^* x_j} \right) \\
				 & = \tr_{L\Gamma}\left( \sum_{i,j=1}^n{
				 \left( \overline{x}_i x_j \right)
				\left( \lambda_{g_i} \lambda_{g_j}^* \right)
				\varphi(\lambda_{g_i}^* \lambda_{g_j})} \right)
			\end{align*}
		\end{proof}
	\item
		\begin{enumerate}
			\item 
			\item 
		\end{enumerate}
	\item
		\begin{enumerate}
			\item 
			\item 
			\item 
			\item 
			\item 
		\end{enumerate}
\end{enumerate}

\end{document}          
