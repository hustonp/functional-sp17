\documentclass[a4paper,10pt]{report}
\usepackage[utf8]{inputenc}
\usepackage{amsmath}
\usepackage{amssymb}
\usepackage{amsthm}
\usepackage{mathtools}
\usepackage{fancyhdr}
\usepackage{enumitem}
\usepackage[top=1in,left=1in,right=1in]{geometry}
\usepackage{mathrsfs}
\usepackage{bm}

\usepackage{bbm}
\usepackage{tikz-cd}
\usepackage{stackengine}
\usepackage{Math_Symbols} 
\usepackage{ifpdf}
\ifpdf
%\usepackage[pdftex]{graphicx}
%\else
%\usepackage[dvips]{graphicx}
%\fi

\setenumerate{listparindent=\parindent,topsep=\parskip}
% \setlist[enumerate]{topsep=\parskip}
\setlist[enumerate,2]{label=(\arabic*)}
\setlist[enumerate,3]{label=(\alph*)}

% \newcommand{\set}[1]{{\{#1\}}}
\newcommand{\ggen}[1]{\langle#1\rangle}
\newcommand{\pn}[2]{||#1||_{#2}}
\newcommand{\bpn}[2]{\left|\left|#1\right|\right|_{#2}}
\newcommand{\norm}[1]{||#1||}
\newcommand{\bnorm}[1]{\left|\left|#1\right|\right|}
\DeclarePairedDelimiter{\ceil}{\lceil}{\rceil}
\DeclarePairedDelimiter{\floor}{\lfloor}{\rfloor}
\DeclarePairedDelimiter{\set}{\{}{\}}
\DeclarePairedDelimiter{\abs}{|}{|}
\DeclarePairedDelimiter{\ket}{|}{\rangle}
\DeclarePairedDelimiter{\bra}{\langle}{|}

\newcommand{\ol}[1]{\overline{#1}}

\renewcommand{\mod}{\text{ mod }}

\renewcommand{\O}{\operatorname{O}} % Bound otherwise
\renewcommand{\o}{\operatorname{o}}
\newcommand{\T}{\text{yes}}
\newcommand{\F}{\text{no}}

\newcommand{\Z}{\mathbb{Z}}
\newcommand{\N}{\mathbb{N}}
\newcommand{\C}{\mathbb{C}}
\newcommand{\Q}{\mathbb{Q}}
\newcommand{\tens}{\otimes}
\newcommand{\rar}[2][]{\overset{#2}{\underset{#1}{\longrightarrow}}}

\DeclareMathOperator{\img}{img}
\DeclareMathOperator{\fop}{int}
\DeclareMathOperator{\fcl}{cl}
% \DeclareMathOperator{\lg}{lg}
\DeclareMathOperator{\vspan}{span}
\DeclareMathOperator{\rng}{rng}
\DeclareMathOperator{\Rng}{Rng}
\DeclareMathOperator{\Cov}{Cov}
\DeclareMathOperator{\Var}{Var}
\DeclareMathOperator{\Bernoulli}{Bernoulli}
\DeclareMathOperator{\Normal}{Normal}
\DeclareMathOperator{\Uniform}{Uniform}
\DeclareMathOperator{\Binom}{Binomial}
\DeclareMathOperator{\mgf}{mgf}
\DeclareMathOperator{\Supp}{Supp}

\newcommand{\cat}[1]{(\bm{#1})}

\providecommand{\Alpha}{A}

\newtheorem*{lemma*}{Lemma}

% Roman numerals
\makeatletter
\newcommand{\Romnum}[1]{\expandafter\@slowromancap\romannumeral #1@}
\makeatother
\newcommand{\factor}[1]{\text{\Romnum{#1}}}

\begin{document}
% \maketitle

\pagestyle{fancy}	
\fancyhf{} % Reset headers and footers
\lhead{NAMES GO HERE\\
Functional Analysis 2\\
\today}
\setlength{\headheight}{60pt}

\begin{center}
	\textbf{Homework 13}
\end{center}
\begin{enumerate}
		\setcounter{enumi}{108}
	\item
		\begin{enumerate}
			\item 
			\item 
			\item 
		\end{enumerate}
	\item 

		Now, we explain why the tensor product of positive matrices valued in a von Neumann algebra is positive. Let $M\subseteq B(H)$ be a von Neumann algebra. We have already established an inclusion $M_n(B(H))\to B(H^n)$, such that if $A,B\in M_n(B(H))$, then the Kronecker tensor product of matrices $A\tens B$ defines the same operator as the operator tensor product $A\tens B$. The operator tensor product is $(A\tens I)(I\tens B)$, so it suffices to prove that the product of two commuting positive elements of a von Neumann algebra is positive. I assume we learned this last semester, so I looked up some of the details on Stack Exchange and in Analysis Now. 
		
		Let $A,B\in M$ be commuting positive operators. The product of commuting normal elements is normal, so it suffices to show that $\spec(AB)\subseteq\mathbb{R}$. We claim that $\Spec(AB)\subseteq\Spec(A)\Spec(B)$. If we let $N=\set{A,B}''$, then $N$ is a commuting von Neumann algebra, so by taking the Gelfand transform, $\Spec_N(B)\subseteq\Spec_N(A)\Spec_N(B)$. Now $\Spec_M\subseteq\Spec_N$, since $M\supseteq N$. However, $\Spec_N(A)=\Spec_M(A)$, because if $A-\lambda I$ is invertible in $A$, then by the continuous functional calculus, the inverse commutes with $A$ and $B$, and is therefore in $N$; similarly for $B$. Therefore, $\Spec_M(AB)\subseteq\Spec_N(AB)\subseteq\Spec_M(A)\Spec_M(B)\subseteq\mathbb{R}$. 
	% \item
	% 	\begin{enumerate}
	% 		\item 
	% 		\item 
	% 	\end{enumerate}
		\setcounter{enumi}{111}
	\item
		\begin{enumerate}
			\item 
			\item 
			\item 
			\item 
			\item 
		\end{enumerate}
\end{enumerate}

\end{document}          
