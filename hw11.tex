\documentclass[a4paper,10pt]{report}
\usepackage[utf8]{inputenc}
\usepackage{amsmath}
\usepackage{amssymb}
\usepackage{amsthm}
\usepackage{mathtools}
\usepackage{fancyhdr}
\usepackage{enumitem}
\usepackage[top=1in,left=1in,right=1in]{geometry}
\usepackage{mathrsfs}
\usepackage{bm}

\usepackage{bbm}
\usepackage{tikz-cd}
\usepackage{stackengine}
\usepackage{Math_Symbols} 
\usepackage{ifpdf}
\ifpdf
%\usepackage[pdftex]{graphicx}
%\else
%\usepackage[dvips]{graphicx}
%\fi

\setenumerate{listparindent=\parindent,topsep=\parskip}
% \setlist[enumerate]{topsep=\parskip}
\setlist[enumerate,2]{label=(\arabic*)}
\setlist[enumerate,3]{label=(\alph*)}

% \newcommand{\set}[1]{{\{#1\}}}
\newcommand{\ggen}[1]{\langle#1\rangle}
\newcommand{\pn}[2]{||#1||_{#2}}
\newcommand{\bpn}[2]{\left|\left|#1\right|\right|_{#2}}
\newcommand{\norm}[1]{||#1||}
\newcommand{\bnorm}[1]{\left|\left|#1\right|\right|}
\DeclarePairedDelimiter{\ceil}{\lceil}{\rceil}
\DeclarePairedDelimiter{\floor}{\lfloor}{\rfloor}
\DeclarePairedDelimiter{\set}{\{}{\}}
\DeclarePairedDelimiter{\abs}{|}{|}
\DeclarePairedDelimiter{\ket}{|}{\rangle}
\DeclarePairedDelimiter{\bra}{\langle}{|}

\newcommand{\ol}[1]{\overline{#1}}

\renewcommand{\mod}{\text{ mod }}

\renewcommand{\O}{\operatorname{O}} % Bound otherwise
\renewcommand{\o}{\operatorname{o}}
\newcommand{\T}{\text{yes}}
\newcommand{\F}{\text{no}}

\newcommand{\Z}{\mathbb{Z}}
\newcommand{\N}{\mathbb{N}}
\newcommand{\C}{\mathbb{C}}
\newcommand{\Q}{\mathbb{Q}}
\newcommand{\textns}{\otimes}
\newcommand{\rar}[2][]{\overset{#2}{\underset{#1}{\longrightarrow}}}

\DeclareMathOperator{\img}{img}
\DeclareMathOperator{\fop}{int}
\DeclareMathOperator{\fcl}{cl}
% \DeclareMathOperator{\lg}{lg}
\DeclareMathOperator{\vspan}{span}
\DeclareMathOperator{\rng}{rng}
\DeclareMathOperator{\Rng}{Rng}
\DeclareMathOperator{\Cov}{Cov}
\DeclareMathOperator{\Var}{Var}
\DeclareMathOperator{\Bernoulli}{Bernoulli}
\DeclareMathOperator{\Normal}{Normal}
\DeclareMathOperator{\Uniform}{Uniform}
\DeclareMathOperator{\Binom}{Binomial}
\DeclareMathOperator{\mgf}{mgf}
\DeclareMathOperator{\Supp}{Supp}

\newcommand{\cat}[1]{(\bm{#1})}

\providecommand{\Alpha}{A}

\newtheorem*{lemma*}{Lemma}

% Roman numerals
\makeatletter
\newcommand{\Romnum}[1]{\expandafter\@slowromancap\romannumeral #1@}
\makeatother
\newcommand{\factor}[1]{\text{\Romnum{#1}}}

\begin{document}
% \maketitle

\pagestyle{fancy}	
\fancyhf{} % Reset headers and footers
\lhead{Ethan Ackelsberg, Zachery Dell, Peter Huston\\
Functional Analysis 2\\
\today}
\setlength{\headheight}{60pt}

\begin{center}
	\textbf{Homework 11}
\end{center}
\begin{enumerate}
		\setcounter{enumi}{102}
	\item
		\begin{enumerate}
			\item First, recall that any Banach space is weak-$*$ dense in its double dual. In particular, $\ell^1\Gamma$ is weak-$*$ dense in $(\ell^\infty\Gamma)^*$. Let $\phi\in(\ell^\infty\Gamma)$ be a positive operator, and pick some $\phi_\lambda\in\ell^1\Gamma$ with $\phi_\lambda\to\phi$ in the weak-$*$ topology. As in class, for $\psi\in\ell^1\Gamma$, let $|\psi|$ be defined by $|\psi|(g)=|\psi(g)|$, and notice that $|\psi|$ is still in $\ell^1$. Therefore, we may replace each $\phi_\lambda$ with $|\phi_\lambda|$, showing that $\phi$ is the weak-* limit of positive members of $\ell^1\Gamma$. Finally, suppose that $\phi$ is a state, i.e. $\phi(M_1)=1$. Define $\psi_\lambda$ by $\psi_\lambda=\frac{|\phi_\lambda|}{|\phi_\lambda(1)|}$. Since $\phi_\lambda(1)\to\phi(1)=1$ already, by passing to a subnet, we may assume that $|\phi_\lambda(1)|\to 1$, so that $\psi_\lambda\to\phi$ as well. Since $\ell^1\Gamma$ is a \csa, this implies that each $\psi_\lambda$ is a state, so we are done. 
			\item 
			\item 
		\end{enumerate}
	\item
		\begin{enumerate}
		\item Without loss of generality we may assume $a \leq b$, and then we have

                  \begin{align*}
                    &\I_0^1 \MG \chi_{\E r , 1 \FD} \E a \R - \chi_{\E r , 1 \FD} \E b \R \GM dr\\
                    =& \I_0^a \MG \chi_{\E r , 1 \FD} \E a \R - \chi_{\E r , 1 \FD} \E b \R \GM dr + \I_a^b \MG \chi_{\E r , 1 \FD} \E a \R - \chi_{\E r , 1 \FD} \E b \R \GM dr + \I_b^1 \MG \chi_{\E r , 1 \FD} \E a \R - \chi_{\E r , 1 \FD} \E b \R \GM dr\\
                    =& \I_a^b dr\\
                    =& b - a
                  \end{align*}
			\item 
			\item 
			\item 
			\item 
			\item 
			\item 
		\end{enumerate}
	\item
		\begin{enumerate}
			\item 
			\item 
			\item 
			\item 
			\item 
		\end{enumerate}
\end{enumerate}

\end{document}          
