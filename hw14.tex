\documentclass[a4paper,10pt]{report}
\usepackage[utf8]{inputenc}
\usepackage{amsmath}
\usepackage{amssymb}
\usepackage{amsthm}
\usepackage{mathtools}
\usepackage{fancyhdr}
\usepackage{enumitem}
\usepackage[top=1in,left=1in,right=1in]{geometry}
\usepackage{mathrsfs}
\usepackage{bm}

\usepackage{bbm}
\usepackage{tikz-cd}
\usepackage{stackengine}
\usepackage{Math_Symbols} 
\usepackage{ifpdf}
\ifpdf
%\usepackage[pdftex]{graphicx}
%\else
%\usepackage[dvips]{graphicx}
%\fi

\setenumerate{listparindent=\parindent,topsep=\parskip}
% \setlist[enumerate]{topsep=\parskip}
\setlist[enumerate,2]{label=(\arabic*)}
\setlist[enumerate,3]{label=(\alph*)}

% \newcommand{\set}[1]{{\{#1\}}}
\newcommand{\ggen}[1]{\langle#1\rangle}
\newcommand{\pn}[2]{||#1||_{#2}}
\newcommand{\bpn}[2]{\left|\left|#1\right|\right|_{#2}}
\newcommand{\norm}[1]{||#1||}
\newcommand{\bnorm}[1]{\left|\left|#1\right|\right|}
\DeclarePairedDelimiter{\ceil}{\lceil}{\rceil}
\DeclarePairedDelimiter{\floor}{\lfloor}{\rfloor}
\DeclarePairedDelimiter{\set}{\{}{\}}
\DeclarePairedDelimiter{\abs}{|}{|}
\DeclarePairedDelimiter{\ket}{|}{\rangle}
\DeclarePairedDelimiter{\bra}{\langle}{|}

\newcommand{\ol}[1]{\overline{#1}}

\renewcommand{\mod}{\text{ mod }}

\renewcommand{\O}{\operatorname{O}} % Bound otherwise
\renewcommand{\o}{\operatorname{o}}
\newcommand{\T}{\text{yes}}
\newcommand{\F}{\text{no}}

\newcommand{\Z}{\mathbb{Z}}
\newcommand{\N}{\mathbb{N}}
\newcommand{\C}{\mathbb{C}}
\newcommand{\Q}{\mathbb{Q}}
\newcommand{\textns}{\otimes}
\newcommand{\rar}[2][]{\overset{#2}{\underset{#1}{\longrightarrow}}}

\DeclareMathOperator{\img}{img}
\DeclareMathOperator{\fop}{int}
\DeclareMathOperator{\fcl}{cl}
% \DeclareMathOperator{\lg}{lg}
\DeclareMathOperator{\vspan}{span}
\DeclareMathOperator{\rng}{rng}
\DeclareMathOperator{\Rng}{Rng}
\DeclareMathOperator{\Cov}{Cov}
\DeclareMathOperator{\Var}{Var}
\DeclareMathOperator{\Bernoulli}{Bernoulli}
\DeclareMathOperator{\Normal}{Normal}
\DeclareMathOperator{\Uniform}{Uniform}
\DeclareMathOperator{\Binom}{Binomial}
\DeclareMathOperator{\mgf}{mgf}
\DeclareMathOperator{\Supp}{Supp}

\newcommand{\cat}[1]{(\bm{#1})}

\providecommand{\Alpha}{A}

\newtheorem*{lemma*}{Lemma}

% Roman numerals
\makeatletter
\newcommand{\Romnum}[1]{\expandafter\@slowromancap\romannumeral #1@}
\makeatother
\newcommand{\factor}[1]{\text{\Romnum{#1}}}

\begin{document}
% \maketitle

\pagestyle{fancy}	
\fancyhf{} % Reset headers and footers
\lhead{NAMES GO HERE\\
Functional Analysis 2\\
\today}
\setlength{\headheight}{60pt}

\begin{center}
	\textbf{Homework 14}
\end{center}
\begin{enumerate}
		\setcounter{enumi}{113}
	\item Suppose $\Gamma$ is a countable group and $(H, \pi)$ is a unitary representation
		on a separable Hilbert space.
		Find a unitary $u \in B(\ell^2\Gamma \ol{\otimes} H)$ intertwining $\lambda \otimes \pi$
		and $\lambda \otimes 1$, i.e. $u(\lambda_g \otimes \pi_g) = (\lambda_g \otimes 1)u$
		for all $g \in \Gamma$.
		\begin{proof}
			Define $u \in B(\ell^2\Gamma \ol{\otimes} H)$ by
			$u(\delta_g \otimes \xi) := \delta_g \otimes \pi_{g^{-1}}\xi$
			for $g \in \Gamma$ and $\xi \in H$.
			This defines a unitary operator, since $\pi$ is a unitary representation.
			We will check that $u$ intertwines $\lambda \otimes \pi$ and $\lambda \otimes 1$.
			For $h \in \Gamma$ and $\xi \in H$, we have
			\begin{align*}
				u(\lambda_g \otimes \pi_g)(\delta_h \otimes \xi)
				 & = u(\delta_{gh} \otimes \pi_g\xi) \\
				 & = \delta_{gh} \otimes \pi_{h^{-1}}\xi \\
				 & = (\lambda_g \otimes 1)(\delta_h \otimes \pi_{h^{-1}}\xi) \\
				 & = (\lambda_g \otimes 1)u(\delta_h \otimes \xi). \\
			\end{align*}
		\end{proof}
	\item 
		\begin{enumerate}
			\item 
			\item 
			\item 
			\item 
			\item 
			\item 
			\item 
		\end{enumerate}
	\item 
\end{enumerate}

\end{document}          
