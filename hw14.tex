\documentclass[a4paper,10pt]{report}
\usepackage[utf8]{inputenc}
\usepackage{amsmath}
\usepackage{amssymb}
\usepackage{amsthm}
\usepackage{mathtools}
\usepackage{fancyhdr}
\usepackage{enumitem}
\usepackage[top=1in,left=1in,right=1in]{geometry}
\usepackage{mathrsfs}
\usepackage{bm}

\usepackage{bbm}
\usepackage{tikz-cd}
\usepackage{stackengine}
\usepackage{Math_Symbols} 
\usepackage{ifpdf}
\ifpdf
%\usepackage[pdftex]{graphicx}
%\else
%\usepackage[dvips]{graphicx}
%\fi

\setenumerate{listparindent=\parindent,topsep=\parskip}
% \setlist[enumerate]{topsep=\parskip}
\setlist[enumerate,2]{label=(\arabic*)}
\setlist[enumerate,3]{label=(\alph*)}

% \newcommand{\set}[1]{{\{#1\}}}
\newcommand{\ggen}[1]{\langle#1\rangle}
\newcommand{\pn}[2]{||#1||_{#2}}
\newcommand{\bpn}[2]{\left|\left|#1\right|\right|_{#2}}
\newcommand{\norm}[1]{||#1||}
\newcommand{\bnorm}[1]{\left|\left|#1\right|\right|}
\DeclarePairedDelimiter{\ceil}{\lceil}{\rceil}
\DeclarePairedDelimiter{\floor}{\lfloor}{\rfloor}
\DeclarePairedDelimiter{\set}{\{}{\}}
\DeclarePairedDelimiter{\abs}{|}{|}
\DeclarePairedDelimiter{\ket}{|}{\rangle}
\DeclarePairedDelimiter{\bra}{\langle}{|}

\newcommand{\ol}[1]{\overline{#1}}

\renewcommand{\mod}{\text{ mod }}

\renewcommand{\O}{\operatorname{O}} % Bound otherwise
\renewcommand{\o}{\operatorname{o}}
\newcommand{\T}{\text{yes}}
\newcommand{\F}{\text{no}}

\newcommand{\Z}{\mathbb{Z}}
\newcommand{\N}{\mathbb{N}}
\newcommand{\C}{\mathbb{C}}
\newcommand{\Q}{\mathbb{Q}}
\newcommand{\textns}{\otimes}
\newcommand{\rar}[2][]{\overset{#2}{\underset{#1}{\longrightarrow}}}

\DeclareMathOperator{\img}{img}
\DeclareMathOperator{\fop}{int}
\DeclareMathOperator{\fcl}{cl}
% \DeclareMathOperator{\lg}{lg}
\DeclareMathOperator{\vspan}{span}
\DeclareMathOperator{\rng}{rng}
\DeclareMathOperator{\Rng}{Rng}
\DeclareMathOperator{\Cov}{Cov}
\DeclareMathOperator{\Var}{Var}
\DeclareMathOperator{\Bernoulli}{Bernoulli}
\DeclareMathOperator{\Normal}{Normal}
\DeclareMathOperator{\Uniform}{Uniform}
\DeclareMathOperator{\Binom}{Binomial}
\DeclareMathOperator{\mgf}{mgf}
\DeclareMathOperator{\Supp}{Supp}

\newcommand{\cat}[1]{(\bm{#1})}

\providecommand{\Alpha}{A}

\newtheorem*{lemma*}{Lemma}

% Roman numerals
\makeatletter
\newcommand{\Romnum}[1]{\expandafter\@slowromancap\romannumeral #1@}
\makeatother
\newcommand{\factor}[1]{\text{\Romnum{#1}}}

\begin{document}
% \maketitle

\pagestyle{fancy}	
\fancyhf{} % Reset headers and footers
\lhead{Ethan Ackelsberg, Zackery Dell, Peter Huston\\
Functional Analysis 2\\
\today}
\setlength{\headheight}{60pt}

\begin{center}
	\textbf{Homework 14}
\end{center}
\begin{enumerate}
		\setcounter{enumi}{113}
	\item Suppose $\Gamma$ is a countable group and $(H, \pi)$ is a unitary representation
		on a separable Hilbert space.
		Find a unitary $u \in B(\ell^2\Gamma \ol{\otimes} H)$ intertwining $\lambda \otimes \pi$
		and $\lambda \otimes 1$, i.e. $u(\lambda_g \otimes \pi_g) = (\lambda_g \otimes 1)u$
		for all $g \in \Gamma$.
		\begin{proof}
			Define $u \in B(\ell^2\Gamma \ol{\otimes} H)$ by
			$u(\delta_g \otimes \xi) := \delta_g \otimes \pi_{g^{-1}}\xi$
			for $g \in \Gamma$ and $\xi \in H$.
			This defines a unitary operator, since $\pi$ is a unitary representation.
			We will check that $u$ intertwines $\lambda \otimes \pi$ and $\lambda \otimes 1$.
			For $h \in \Gamma$ and $\xi \in H$, we have
			\begin{align*}
				u(\lambda_g \otimes \pi_g)(\delta_h \otimes \xi)
				 & = u(\delta_{gh} \otimes \pi_g\xi) \\
				 & = \delta_{gh} \otimes \pi_{h^{-1}}\xi \\
				 & = (\lambda_g \otimes 1)(\delta_h \otimes \pi_{h^{-1}}\xi) \\
				 & = (\lambda_g \otimes 1)u(\delta_h \otimes \xi). \\
			\end{align*}
		\end{proof}
	\item 
		\begin{enumerate}
				\setcounter{enumii}{1}
			\item In the notes, the measure on $\mathcal{R}$ induced by $\mu$ is defined to be $\nu=\theta_*(\mu\times\gamma)$. Therefore, $\theta$ is certainly a bijective isomorphism of measure spaces $X\times\gamma\to\mathcal{R}$. 
				\setcounter{enumii}{0}
			\item For $\eta\in L^2\mathcal{R}$ and $(x,g)\in X\times\Gamma$, define $v\eta(x,g)=\eta\circ\theta$. Then for all $\eta,\xi\in L^2\mathcal{R}$, we have $\int(v\eta)(v\xi)d(\mu\times\gamma)=\int\eta\circ\theta\xi\circ\theta d(\mu\times\gamma)=\int\eta\xi d\nu$, so $v$ is an isometry. Since $\theta$ is a bijection, $v^*\eta=\eta\circ\theta^{-1}$, and so $vv^*=1_{L^2(X\times\Gamma)}$ and $v^*v=1_{L^2\mathcal{R}}$. 
				\setcounter{enumii}{2}
			\item Let $\eta\in L^2\mathcal{R}$ and $(x,g)\in X\times\Gamma$. We compute:
				\begin{align*}
					M_fv\eta(x,g) &= f(x)v\eta(x,g)\\
					&= f(x)\eta(x,g^{-1}x)\\
					&= \lambda(f)\eta(x,g^{-1}x)\\
					&= v\lambda(f)\eta(x,g)
				\end{align*}
			\item Again, we have 
				\begin{align*}
					u_gv\eta(x,h) &= v\eta(g^{-1}x,g^{-1}h)\\
					&= \eta(g^{-1}x,h^{-1}x)\\
					&= \chi_{gX}(x)\eta(\varphi_g^{-1}(x),h^{-1}x)\\
					&= L_{\varphi_g}\eta(x,h^{-1}x)\\
					&= uL_{\varphi_g}\eta(x,h)
				\end{align*}
			\item % Say why $\lambda(L^\infty)\subseteq L\mathcal{R}$?
				Because $L^\infty(X,\mu)\unlhd\Gamma=(\set{u_g}\bigcup\set{M_f})''$ and $v$ is unitary, we have $v^*(L^\infty(X,\mu)\unlhd\Gamma)v\subseteq L\mathcal{R}''$. Since $L\mathcal{R}$ is a von Neumann algebra, it is already its own bicommutant, giving the desired containment. 
			\item The point of problems 92 and 93 was to show that the commutant of $L^\infty(X,\mu)\unlhd\Gamma$ is itself a kind of right semidirect product, generated by the right multipliers $(W_f)_{f\in L^\infty}$ and $(n_g)_{g\in\Gamma}$, with $W_fv\eta(x,g)=f(g^{-1}x)\eta(x,g)$ and $n_gv\eta(x,h)=v\eta(x,hg)$. % Strangely, the right multipliers n_g define a left action. 
				Therefore, it suffices to show that for all $f\in L^\infty$ and $g\in\Gamma$, we have $v^*W_fv,v^*n_gv\in R\mathcal{R}$. The proof is analagous to the previous two parts. Observe: 
				\begin{align*}
					W_fv\eta(x,g) &= f(g^{-1}x)v\eta(x,g)\\
					&= f(g^{-1}(x))\eta(x,g^{-1}(x))\\
					&= \rho(f)\eta(x,g^{-1}x)\\
					&= v\rho(f)\eta(x,g)
				\end{align*}
				and 
				\begin{align*}
					n_gv\eta(x,h) &= v\eta(x,hg)\\
					&= \eta(x,g^{-1}h^{-1}x)\\
					&= \eta(x,\varphi_g^{-1}(h^{-1}(x)))\\
					&= R_{\varphi_g}\eta(x,h^{-1}x)\\
					&= vR_{\varphi_g}(x,h)
				\end{align*}
			\item Analagous to (5), we obtain $(v^*(L^\infty(X,\mu)\unlhd\Gamma)v)'=v^*(L^\infty(X,\mu)\unlhd\Gamma)'v\subseteq R\mathcal{R}\subseteq L\mathcal{R}'$, so $v^*(L^\infty(X,\mu)\unlhd\Gamma)v\supseteq L\mathcal{R}$, completing the proof. 
		\end{enumerate}





              \item For each $f \in L^\infty \E X , \mu \R \subs L \mathcal R$, we have

                \begin{align*}
                  \E f \cdot \xi \R \E x , y \R = f \E x \R \xi \E x , y \R
                \end{align*}
                and for $g \in J A J \subs L \mathcal R' = R \mathcal R$ we have
                \begin{align*}
                  \E g \cdot \xi \R \E x , y \R = g \E y \R \xi \E x , y \R.
                \end{align*}
                Then we have $A \subs L^\infty \mathcal R$ and $J A J \subs L^\infty \mathcal R$, by viewing $f \in A$ as $f \otimes 1 \in L^\infty \mathcal R$, and viewing $g \in J A J$ as $1 \otimes g \in L^\infty \mathcal R$.
                Thus $A \cup J A J \subs L^\infty \E \mathcal R \R$, and now we show the other inclusion. Since both are SOT-closed, it suffices to show that $A \cup J A J$ is SOT-dense in $L^\infty \mathcal R$.
                To do this we just need to show that the linear span of functions of the form $\E f \otimes 1 \R$ $\E 1 \otimes g \R$ for $f , g \in L^\infty \E X , \mu \R$ are SOT-dense in $L^\infty \mathcal R$.
                Linear combinations of indicators are norm-dense in $L^\infty$, so it suffices to show that the linear span of indicators of rectangles is SOT-dense in indicators of $\mathcal R$-measurable sets. 

																In the notation of the last problem, we have $v^*L^\infty(\mathcal{R},\nu)v=L^\infty(X\times\Gamma,\mu\times\gamma)$. There, up to null sets, all measureable sets are the countable union of measurable rectangles, since $\Gamma$ is countable. In particular, if $S\subseteq\mathcal{R}$ and $T_n$ is a sequence of indicator functions converging pointwise to $\chi_{\theta^{-1}(S)}$ from below, then for every $f\in L^2(\mu\times\gamma)$, we have $|f|^2d\nu$ a finite measure absolutely continuous with respect to $\nu$, so $\norm{T_nf}_2^2\to\norm{\chi_{\theta^{-1}(S)}f}_2^2$. Finally, if $R\subseteq X\times\Gamma$ is a rectangle, then so is $\theta(R)$, so the linear span of indicators of rectangles is also SOT-dense in $L^\infty\mathcal{R}$, as desired. 
\end{enumerate}

\end{document}          
