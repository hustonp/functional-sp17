\documentclass[a4paper,10pt]{report}
\usepackage[utf8]{inputenc}
\usepackage{amsmath}
\usepackage{amssymb}
\usepackage{amsthm}
\usepackage{mathtools}
\usepackage{fancyhdr}
\usepackage{enumitem}
\usepackage[top=1in,left=1in,right=1in]{geometry}
\usepackage{mathrsfs}
\usepackage{bm}

\usepackage{bbm}
\usepackage{tikz-cd}
\usepackage{stackengine}
\usepackage{Math_Symbols} 
\usepackage{ifpdf}
\ifpdf
%\usepackage[pdftex]{graphicx}
%\else
%\usepackage[dvips]{graphicx}
%\fi

\setenumerate{listparindent=\parindent,topsep=\parskip}
% \setlist[enumerate]{topsep=\parskip}
\setlist[enumerate,2]{label=(\arabic*)}
\setlist[enumerate,3]{label=(\alph*)}

% \newcommand{\set}[1]{{\{#1\}}}
\newcommand{\ggen}[1]{\langle#1\rangle}
\newcommand{\pn}[2]{||#1||_{#2}}
\newcommand{\bpn}[2]{\left|\left|#1\right|\right|_{#2}}
\newcommand{\norm}[1]{||#1||}
\newcommand{\bnorm}[1]{\left|\left|#1\right|\right|}
\DeclarePairedDelimiter{\ceil}{\lceil}{\rceil}
\DeclarePairedDelimiter{\floor}{\lfloor}{\rfloor}
\DeclarePairedDelimiter{\set}{\{}{\}}
\DeclarePairedDelimiter{\abs}{|}{|}
\DeclarePairedDelimiter{\ket}{|}{\rangle}
\DeclarePairedDelimiter{\bra}{\langle}{|}

\newcommand{\ol}[1]{\overline{#1}}

\renewcommand{\mod}{\text{ mod }}

\renewcommand{\O}{\operatorname{O}} % Bound otherwise
\renewcommand{\o}{\operatorname{o}}
\newcommand{\T}{\text{yes}}
\newcommand{\F}{\text{no}}

\newcommand{\Z}{\mathbb{Z}}
\newcommand{\N}{\mathbb{N}}
\newcommand{\C}{\mathbb{C}}
\newcommand{\Q}{\mathbb{Q}}
\newcommand{\textns}{\otimes}
\newcommand{\rar}[2][]{\overset{#2}{\underset{#1}{\longrightarrow}}}

\DeclareMathOperator{\img}{img}
\DeclareMathOperator{\fop}{int}
\DeclareMathOperator{\fcl}{cl}
% \DeclareMathOperator{\lg}{lg}
\DeclareMathOperator{\vspan}{span}
\DeclareMathOperator{\rng}{rng}
\DeclareMathOperator{\Rng}{Rng}
\DeclareMathOperator{\Cov}{Cov}
\DeclareMathOperator{\Var}{Var}
\DeclareMathOperator{\Bernoulli}{Bernoulli}
\DeclareMathOperator{\Normal}{Normal}
\DeclareMathOperator{\Uniform}{Uniform}
\DeclareMathOperator{\Binom}{Binomial}
\DeclareMathOperator{\mgf}{mgf}
\DeclareMathOperator{\Supp}{Supp}

\newcommand{\cat}[1]{(\bm{#1})}

\providecommand{\Alpha}{A}

\newtheorem*{lemma*}{Lemma}

% Roman numerals
\makeatletter
\newcommand{\Romnum}[1]{\expandafter\@slowromancap\romannumeral #1@}
\makeatother
\newcommand{\factor}[1]{\text{\Romnum{#1}}}

\begin{document}
% \maketitle

\pagestyle{fancy}	
\fancyhf{} % Reset headers and footers
\lhead{Ethan Ackelsberg, Zachery Dell, Peter Huston\\
Functional Analysis 2\\
\today}
\setlength{\headheight}{60pt}

\begin{center}
	\textbf{Homework 8}
\end{center}

\begin{enumerate}
		\setcounter{enumi}{88}
	\item Let $M\subseteq B(H)$ and $N\subseteq B(K)$. 
		\begin{enumerate}
			\item For a bounded net of positive operators $x_\lambda$ increasing to a positive operator $x$, we know that convergence in the WOT, SOT, $\sigma$-WOT, and $\sigma$-SOT are all equivalent. If $\Phi$ is $\sigma$-WOT continuous, then in particular, if $x\lambda\to x$ in the $\sigma$-WOT, then $\Phi(x_\lambda)\to\Phi(x)$ in the $\sigma$-WOT as well. On the other hand, suppose that $\Phi:M\to N$ is normal. Let $\psi\in N_*$, the dual of $N$ under the $\sigma$-WOT topology, be positive. Since $\psi$ is $\sigma$-WOT continuous, $\psi$ is normal, so $\psi\Phi$ is normal and positive, and hence $\sigma$-WOT continuous. Since $M_*$ is spanned by positive linear functionals (Corollary 4.3.4 of Jesse Peterson's notes), % HERE expand?
				this shows that $\psi\Phi$ is continuous for every $\psi\in N_*$. Since the $\sigma$-WOT topology is the weak topology induced by the predual, this is precisely the condition for $\Phi$ to be $\sigma$-WOT continuous. 
			\item If $\ker(\Phi)=0$, then by problem 72 part 2, we know that $\Phi(M)$ is a von-Neumann subalgebra of $N$. Therefore, it suffices to construct  unital Von-Neumann algebra $*$-homomorphism $C:M\to\coker(\Phi)$, with an injective factorization $\ol{\Phi}:\coker(\Phi)\to N$ so that $\Phi=\ol{\Phi}C$. Since $0\subseteq N$ is $\sigma$-WOT closed and $\Phi$ is $\sigma$-WOT continuous, $\ker(\Phi)$ is $\sigma$-WOT closed. Since $\Phi$ is a ring homomorphism, $\ker(\Phi)$ is a $2$-sided ideal. Therefore, by results proven in class, $\ker(\Phi)$ is of the form $Mz$ for some $z\in P(Z(M))$. Since $z\in Z(M)$, we know that $zH$ and $(1-z)H$ are $M$-invariant subspaces of $H$; if $m-n\in\ker(\Phi)$, then $(m-n)z=m-n$, so $(m-n)(1-z)=0$. Therefore, let $C:M\to M(1-z)$ be the compression map, a $\sigma$-WOT continuous unital $*$-homomorphism for sure. If $C(x)=C(y)$, then $\Phi(x)=\Phi(y)$, so the map $\ol{\Phi}:M(1-z)\to N$ given by $\ol{\Phi}(m(1-z))=\Phi(m)$ is a well-defined homomorphism. By definition, $\ol{\Phi}$ is unital, and since $(1-z)$ is central and self-adjoint, $C$ and $\ol{\Phi}$ are $*$-homomorphisms. But $(m(1-z))(1-z)=m(1-z)$, so $\ol{\Phi}$ is actually a restriction of $\Phi$, and hence still $\sigma$-WOT continuous. 

				For the sake of completeness, we should say why the $\sigma$-WOT topology on $M(1-z)$ is the subspace topology coming from $M$. When constructing the predual, we found that for a von-Neumann algebra $A$, we have $A_*\cong L^1(B(H))/A_\perp$, where $A_\perp$ is precisely those elements $y\in L^1(B(H))$ so that $\tr(ay)=0$ for every $a\in A$. Therefore, the $\sigma$-WOT topology on any $A\subseteq B(H)$ is actually induced by the (usually redundant) family of seminorms $(a\to |\tr(ay)|)_{y\in L^1(B(H))}$, and every von-Neumann algebra $A\subseteq B(H)$ has the induced $\sigma$-WOT topology from $B(H)$. % HERE true? neccesary to say in HW, or obvious by now?
			\item 
				\begin{enumerate}
					\item 
					\item 
				\end{enumerate}
		\end{enumerate}
		\setcounter{enumi}{90}
		\newpage
 \item 
		\begin{enumerate}
			\item 
			\item 
			\item 
			\item 
			\item 
			\item 
			\item 
			\item 
		\end{enumerate}
		\newpage
 \item 
		\begin{enumerate}
			\item 
			\item 
		\end{enumerate}
\end{enumerate}

\end{document}          
