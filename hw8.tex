\documentclass[a4paper,10pt]{report}
\usepackage[utf8]{inputenc}
\usepackage{amsmath}
\usepackage{amssymb}
\usepackage{amsthm}
\usepackage{mathtools}
\usepackage{fancyhdr}
\usepackage{enumitem}
\usepackage[top=1in,left=1in,right=1in]{geometry}
\usepackage{mathrsfs}
\usepackage{bm}

\usepackage{bbm}
\usepackage{tikz-cd}
\usepackage{stackengine}
\usepackage{Math_Symbols} 
\usepackage{ifpdf}
\ifpdf
%\usepackage[pdftex]{graphicx}
%\else
%\usepackage[dvips]{graphicx}
%\fi

\setenumerate{listparindent=\parindent,topsep=\parskip}
% \setlist[enumerate]{topsep=\parskip}
\setlist[enumerate,2]{label=(\arabic*)}
\setlist[enumerate,3]{label=(\alph*)}

% \newcommand{\set}[1]{{\{#1\}}}
\newcommand{\ggen}[1]{\langle#1\rangle}
\newcommand{\pn}[2]{||#1||_{#2}}
\newcommand{\bpn}[2]{\left|\left|#1\right|\right|_{#2}}
\newcommand{\norm}[1]{||#1||}
\newcommand{\bnorm}[1]{\left|\left|#1\right|\right|}
\DeclarePairedDelimiter{\ceil}{\lceil}{\rceil}
\DeclarePairedDelimiter{\floor}{\lfloor}{\rfloor}
\DeclarePairedDelimiter{\set}{\{}{\}}
\DeclarePairedDelimiter{\abs}{|}{|}
\DeclarePairedDelimiter{\ket}{|}{\rangle}
\DeclarePairedDelimiter{\bra}{\langle}{|}

\newcommand{\ol}[1]{\overline{#1}}

\renewcommand{\mod}{\text{ mod }}

\renewcommand{\O}{\operatorname{O}} % Bound otherwise
\renewcommand{\o}{\operatorname{o}}
\newcommand{\T}{\text{yes}}
\newcommand{\F}{\text{no}}

\newcommand{\Z}{\mathbb{Z}}
\newcommand{\N}{\mathbb{N}}
\newcommand{\C}{\mathbb{C}}
\newcommand{\Q}{\mathbb{Q}}
\newcommand{\textns}{\otimes}
\newcommand{\rar}[2][]{\overset{#2}{\underset{#1}{\longrightarrow}}}

\DeclareMathOperator{\img}{img}
\DeclareMathOperator{\fop}{int}
\DeclareMathOperator{\fcl}{cl}
% \DeclareMathOperator{\lg}{lg}
\DeclareMathOperator{\vspan}{span}
\DeclareMathOperator{\rng}{rng}
\DeclareMathOperator{\Rng}{Rng}
\DeclareMathOperator{\Cov}{Cov}
\DeclareMathOperator{\Var}{Var}
\DeclareMathOperator{\Bernoulli}{Bernoulli}
\DeclareMathOperator{\Normal}{Normal}
\DeclareMathOperator{\Uniform}{Uniform}
\DeclareMathOperator{\Binom}{Binomial}
\DeclareMathOperator{\mgf}{mgf}
\DeclareMathOperator{\Supp}{Supp}

\newcommand{\cat}[1]{(\bm{#1})}

\providecommand{\Alpha}{A}

\newtheorem*{lemma*}{Lemma}

% Roman numerals
\makeatletter
\newcommand{\Romnum}[1]{\expandafter\@slowromancap\romannumeral #1@}
\makeatother
\newcommand{\factor}[1]{\text{\Romnum{#1}}}

\begin{document}
% \maketitle

\pagestyle{fancy}	
\fancyhf{} % Reset headers and footers
\lhead{Ethan Ackelsberg, Zachary Dell, Peter Huston\\
Functional Analysis 2\\
\today}
\setlength{\headheight}{60pt}

\begin{center}
	\textbf{Homework 8}
\end{center}

\begin{enumerate}
		\setcounter{enumi}{88}
	\item Let $\Phi : M \to N$ be a unital $*$-homomorphism between von Neumann algebras.
		\begin{enumerate}
			\item Prove that the following two conditions are equivalent:
			\begin{enumerate}
				\item	$\Phi$ is \emph{normal}: $x_\lambda \nearrow x$
				implies $\Phi(x_\lambda) \nearrow \Phi(x)$.
				\item	$\Phi$ is $\sigma$-WOT continuous.
			\end{enumerate}
			\begin{proof}
			For a bounded net of positive operators $x_\lambda$ increasing to a positive operator $x$, we know that convergence in the WOT, SOT, $\sigma$-WOT, and $\sigma$-SOT are all equivalent. Suppose $\Phi$ is $\sigma$-WOT continuous. If $x_\lambda\nearrow x$, then in particular $x_\lambda\to x$ $\sigma$-WOT, so $\Phi(x_\lambda)\to\Phi(x)$ $\sigma$-WOT as well.
			
			On the other hand, suppose that $\Phi:M\to N$ is normal. Let $\psi\in N_*$, the dual of $N$ under the $\sigma$-WOT topology, be positive. Since $\psi$ is $\sigma$-WOT continuous, $\psi$ is normal, so $\psi \circ \Phi$ is normal and positive, and hence $\sigma$-WOT continuous by problem 88. Since $M_*$ is spanned by positive linear functionals (Corollary 4.3.4 of Jesse Peterson's notes), % HERE expand?
				this shows that $\psi \circ \Phi$ is $\sigma$-WOT continuous for every $\psi\in N_*$. Since the $\sigma$-WOT topology is the weak topology induced by the predual, this is precisely the condition for $\Phi$ to be $\sigma$-WOT continuous. 
				\end{proof}
			\item Prove that if $\Phi$ is normal, then $\Phi(M) \subseteq N$ is a von Neumann subalgebra.
			\begin{proof} If $\ker(\Phi)=0$, then by problem 72 part 2, we know that $\Phi(M)$ is a von-Neumann subalgebra of $N$. Therefore, it suffices to construct  unital Von-Neumann algebra $*$-homomorphism $C:M\to\coker(\Phi)$, with an injective factorization $\ol{\Phi}:\coker(\Phi)\to N$ so that $\Phi=\ol{\Phi}C$. Since $\set{0}\subseteq N$ is $\sigma$-WOT closed and $\Phi$ is $\sigma$-WOT continuous, $\ker(\Phi)$ is $\sigma$-WOT closed. Since $\Phi$ is a ring homomorphism, $\ker(\Phi)$ is a $2$-sided ideal. By results proven in class, $\ker(\Phi)$ is of the form $Mz$ for some $z\in P(Z(M))$. Since $z\in Z(M)$, we know that $zH$ and $(1-z)H$ are $M$-invariant subspaces of $H$; if $m-n\in\ker(\Phi)$, then $(m-n)z=m-n$, so $(m-n)(1-z)=0$. Let $C:M\to M(1-z)$ be the compression map, a $\sigma$-WOT continuous unital $*$-homomorphism for sure. If $C(x)=C(y)$, then $\Phi(x)=\Phi(y)$, so the map $\ol{\Phi}:M(1-z)\to N$ given by $\ol{\Phi}(m(1-z))=\Phi(m)$ is a well-defined homomorphism. By definition, $\ol{\Phi}$ is unital, and since $(1-z)$ is central and self-adjoint, $C$ and $\ol{\Phi}$ are $*$-homomorphisms. But $(m(1-z))(1-z)=m(1-z)$, so $\ol{\Phi}$ is actually a restriction of $\Phi$, and hence still $\sigma$-WOT continuous. 

				For the sake of completeness, we should say why the $\sigma$-WOT topology on $M(1-z)$ is the subspace topology coming from $M$. When constructing the predual, we found that for a von-Neumann algebra $A$, we have $A_*\cong L^1(B(H))/A_\perp$, where $A_\perp$ is precisely those elements $y\in L^1(B(H))$ so that $\tr(ay)=0$ for every $a\in A$. Therefore, the $\sigma$-WOT topology on any $A\subseteq B(H)$ is actually induced by the (usually redundant) family of seminorms ${(a\to |\tr(ay)|)_{y\in L^1(B(H))}}$, and every von-Neumann algebra $A\subseteq B(H)$ has the induced $\sigma$-WOT topology from $B(H)$. % HERE true? neccesary to say in HW, or obvious by now?
				\end{proof}
			\item Let $\varphi$ be a normal state on a von Neumann algebra $M$,
			and let $(H_\varphi, \Omega_\varphi, \pi_\varphi)$ be the cyclic GNS representation
			of $M$ associated to $\varphi$, i.e. $H_\varphi = L^2(M, \varphi)$,
			$\Omega_\varphi \in H_\varphi$ is the image of $1 \in M$ in $H_\varphi$,
			and $\pi_\varphi(x)m\Omega_\varphi = xm\Omega_\varphi$ for all $x, m \in M$.
				\begin{enumerate}
					\item Show that $\pi_\varphi$ is normal.
					\begin{proof}
					Suppose $x_\lambda\to x$ $\sigma$-WOT. For every $z,y\in M$, we have 
						\begin{align*}
							\ggen{\pi_\varphi(x-x_\lambda)y\Omega,z\Omega} &= \varphi(z^*(x-x_\lambda)y)\\
							\intertext{Since multiplication is $\sigma$-WOT continuous in each coordinate and $\varphi$ is $\sigma$-WOT continuous by (1), }
							\ggen{\pi_\varphi(x-x_\lambda)y\Omega,z\Omega}&\to 0
						\end{align*}
							Since $z$ and $y$ were arbitrary and vectors of the form $y\Omega$ are norm-dense in $H_\varphi$, this shows that $\pi_\varphi(x_\lambda)\to\pi_\varphi(x)$ WOT. 
							Since $\pi_\varphi(x_\lambda)$ is bounded by $\norm{\pi_\varphi(x)}$, and the $\sigma$-WOT and WOT agree on bounded sets, $\pi_\varphi(x_\lambda)\to\pi_\varphi(x)$ $\sigma$-WOT as well. % Because of the nature of $\nearrow$, this line is superfluous I guess. 
							By part (1), this is enough to show that $\pi_\varphi$ is normal. 
							\end{proof}
						\item Deduce that if $\varphi$ is faithful,
						then $M \cong \pi_\varphi(M) \subseteq B(H_\varphi)$
						is a von Neumann algebra acting on $H_\varphi$.
						\begin{proof}
						If $\varphi$ is faithful, then for $x\neq y$, we have 
							\begin{align*}
								\norm{\pi_\varphi(x-y)}^2 &\ge \pn{\pi_\varphi(x-y)\Omega}{\varphi}^2\\
								&= \varphi((x-y)^*(x-y))\\
								& > 0,
							 %	\intertext{Because $M$ is a $C^*$-algebra, $\norm{(x-y)(x-y)^*}=\norm{x-y}^2>0$, so } % maybe clearer, but really dumb
							 \end{align*}
							since $(x - y)^*(x- y)$ is a nonzero positive operator.
							Hence, $\pi_\varphi$ is a faithful representation and the claim follows from (2).
							\end{proof}
				\end{enumerate}
		\end{enumerate}
		\setcounter{enumi}{90}
		\newpage
 \item 
		\begin{enumerate}
		\item It's easy to see that $J$ is a conjugate linear isometry: 

                  \begin{align*}
                    J \E \E \lambda a + b \R \Omega \R%
                    &= \E \lambda a + b \R^* \Omega\\
                    &= \ba \lambda a \Omega + b \Omega.
                  \end{align*}

                  and

                  \begin{align*}
                    \NK J \E a \Omega \R \KN%
                    &= \NK a^* \Omega \KN\\
                    &= \ET a^* \Omega , a^* \Omega \TE\\
                    &= \tr \E a a^* \R\\
                    &= \tr \E a^* a \R\\
                    &= \ET a \Omega , a \Omega \TE\\
                    &= \NK a \Omega \KN.
                  \end{align*}

                  Lastly, $J \DF M \Omega \FD = M \Omega$, because $M$ is *-closed.  Since $M \Omega$ is dense in $L^2 M$ by construction, we have that $J \DF M \Omega \FD$ is dense in $L^2 M$ as well.
                \item From the previous part, we know that $J$ extends uniquely to a conjugate linear isometry $L^2 M \to L^2 M$.\\
                  Now fix $\xi \in L^2 M$, and take a sequence $\E a_n \R_{n \in \mathbb N}$ in $M$ such that $a_n \Omega \to \xi$.  Then 

                  \begin{align*}
                    J^2 \xi%
                    &= J^2 \E \lim_{n \to \infty} a_n \Omega \R\\
                    &= \lim_{n \to \infty} J^2 \E a_n \Omega \R, \quad \te{ since $J^2$ is continuous}\\
                    &= \lim_{n \to \infty} \E \E a_n^* \R^* \Omega \R\\
                    &= \lim_{n \to \infty} a_n \Omega\\
                    &= \xi
                  \end{align*}

                  so that $J^2 = 1$ on $L^2 M$.\\
                  Next, for $\xi = b \Omega , \eta = a \Omega \in M \Omega$, we have

                  \begin{align*}
                    \ET J \eta , J \xi \TE%
                    &= \ET a^* \Omega , b^* \Omega\TE\\
                    &= \tr \E b a^* \R\\
                    &= \tr \E a^* b \R\\
                    &= \ET b \Omega , a \Omega \TE\\
                    &= \ET \xi , \eta \TE.
                  \end{align*}

                  Now if $\xi , \eta \in L^2 M$, we can write $\xi = \lim_{n \to \infty} a_n \Omega$ and $\eta = \lim_{n \to \infty} b_n \Omega$ for some sequences $\E a_n \R , \E b_n \R \in M$.\\
                  Then for each $k \in \mathbb N$,

                  \begin{align*}
                    \ET J \eta , J a_k \Omega \TE%
                    &= \ET J \lim_{n \to \infty} b_n \Omega , J a_k \Omega \TE\\
                    &= \lim_{n \to \infty} \ET J \E b_n \Omega \R , J \E a_k \Omega \R \TE\\
                    &= \lim_{n \to \infty} \ET a_k \Omega , b_n \Omega \TE\\
                    &= \ET a_k \Omega , \lim_{n \to \infty} b_n \Omega \TE\\
                    &= \ET a_k \Omega , \eta \TE.
                  \end{align*}

                  Then

                  \begin{align*}
                    \ET J \eta , J \xi \TE%
                    &= \ET J \eta , J \lim_{k \to \infty} a_k \Omega \TE\\
                    &= \lim_{k \to \infty} \ET J \eta , J a_k \Omega \TE\\
                    &= \lim_{k \to \infty} \ET a_k \Omega , \eta \TE\\
                    &= \ET \lim_{k \to \infty} a_k \Omega , \eta \TE\\
                    &= \ET \xi , \eta \TE.
                  \end{align*}

                  Thus $\ET J \eta , J \xi \TE = \ET \xi , \eta \TE$ for all $\xi , \eta \in L^2 M$.\\
                  

		\item For $a , b \in M$, we have

                  \begin{align*}
                    J a^* J b \Omega%
                    &= J a^* b^* \Omega\\
                    &= \E b a \R \Omega,
                  \end{align*}

                  and thus for $a , b , m \in M$,

                  \begin{align*}
                    \E J m J b \R \E a \xi \R%
                    &= \E J m J \R \E b a \R \xi\\
                    &= b a m^* \xi\\
                    &= b \E J m J a \R \xi\\
                    &= b \E J m J \R \E a \xi \R.
                  \end{align*}

                  Thus $J m J$ commutes with $b$ on $M \Omega$, and since $M \Omega$ is dense in $L^2 M$, it follows that $J m J$ commutes with $b$ on $L^2 M$.\\
                  Since $b \in M$ was arbitrary, it follows that $J m J \in M'$, so $J M J \subs M'$.
                  

	        \item For $a , b , c \in M$, we compute

                  \begin{align*}
                    \ET J a^* J b \Omega , c \Omega \TE%
                    &= \ET b a \Omega , c \Omega \TE\\
                    &= \tr \E c^* b a \R\\
                    &= \tr \E a c^* b \R\\
                    &= \ET b \Omega , \E a c^* \R^* \Omega \TE\\
                    &= \ET b \Omega , c a^* \Omega \TE\\
                    &= \ET b \Omega , J a J c \Omega \TE.
                  \end{align*}

                  Thus $\E J a J \R^* = J a^* J$, again because $M \Omega$ is dense in $L^2 M$.


		\item For all $a \in M$ and $y \in M'$, we have

                  \begin{align*}
                    \ET J y \Omega , a \Omega \TE%
                    &= \ET J y \Omega , J a^* \Omega \TE\\
                    &= \ET a^* \Omega , y \Omega \TE\\
                    &= \ET y^* a^* \Omega , \Omega \TE\\
                    &= \ET a^* y \Omega , \Omega \TE\\
                    &= \ET y^* \Omega , a \Omega \TE.
                  \end{align*}

                  Since $\ET J y \Omega , a \Omega \TE = \ET y^* \Omega , a \Omega \TE$ for all $a \in M$, and $M \Omega$ is dense in $L^2 M$, we have $J y \Omega = y^* \Omega$ for all $y \in M'$.

		\item Take $x , y , z \in M'$.  Then

                  \begin{align*}
                    J x^* J y \Omega%
                    &= J x^* y^* \Omega\\
                    &= y x \Omega,
                  \end{align*}

                  and thus

                  \begin{align*}
                    \ET J x^* J y \Omega , z \Omega \TE%
                    &= \ET y x \Omega , z \Omega \TE\\
                    &= \tr \E z^* y x \R\\
                    &= \tr \E x z^* y \R \\
                    &= \ET y \Omega , \E x z^* \R^* \Omega \TE\\
                    &= \ET y \Omega , z x^* \Omega \TE\\
                    &= \ET y \Omega , J x J z \Omega \TE,
                  \end{align*}

                  so that $\E J x J \R^* = J x^* J$.\\

                \item For $a , b \in M$ and $x , y \in M'$, we have

                  \begin{align*}
                    \ET x J y J a \Omega , b \Omega \TE%
                    &= \ET a \Omega , J y^* J x^* b \Omega \TE\\
                    &= \ET a \Omega , J y^* J b x^* \Omega \TE\\
                    &= \tr \E x b^* J y J a \R\\
                    &= \tr \E b^* J y J a x \R\\
                    &= \tr \E b^* J y J x a \R\\
                    &= \ET J y J x a \Omega , b \Omega \TE
                  \end{align*}






                \item Thus from the above we see that each $x \in M'$ commutes with each $J y J \in J M' J$, so that $M' \subs \E J M' J \R' = J M J$, so that $M' = J M J$.\\



		\end{enumerate}
		\newpage
 \item 
 Let $\Gamma$ be a discrete group, and let $L\Gamma = \set{\lambda_g}'' \subseteq B(\ell^2\Gamma)$.
 Consider the faithful $\sigma$-WOT continuous tracial state
 $\tr(x) = \ggen{x\delta_e, \delta_e}$ on $L\Gamma$.
		\begin{enumerate}
			\item Show that $u\delta_g = \lambda_g$ uniquely extends to a unitary
				$u \in B(\ell^2\Gamma, L^2L\Gamma)$ such that for all $x \in L\Gamma$
				and $\xi \in \ell^2\Gamma$, $L_xu\xi = ux\xi$ where $L_x \in B(L^2L\Gamma)$
				is left multiplication by $x$, i.e., $L_x(y\Omega) = xy\Omega$.
				\begin{proof}
					Since $(\delta_g)_{g \in \Gamma}$ is an orthonormal basis in $\ell^2\Gamma$,
					in order to show that $u\delta_g = \lambda_g$ extends uniquely to a unitary,
					it suffices to prove that $(\lambda_g\Omega)_{g \in \Gamma}$ is an
					orthonormal basis for $L^2L\Gamma$.
					First, we check that $(\lambda_g\Omega)_{g \in \Gamma}$ is orthonormal:
					\begin{align*}
						\ggen{\lambda_g\Omega, \lambda_h\Omega}
						 & = \tr(\lambda_h^*\lambda_g) \\
						 & = \ggen{\lambda_g\delta_e, \lambda_h\delta_e} \\
						 & = \ggen{\delta_g, \delta_h} \\
						 & = \delta_{g = h}.
					\end{align*}
					
					It remains to show that $\text{span}\set{\lambda_g\Omega : g \in \Gamma}$
					is dense in $L^2L\Gamma$.
					By construction, $\set{x\Omega : x \in L\Gamma}$ is dense in $L^2L\Gamma$,
					so it suffices to prove that $\text{span}\set{\lambda_g\Omega : g \in \Gamma}$
					is dense in $\set{x\Omega : x \in L\Gamma}$ is dense in $L^2L\Gamma$.
					
					Let $x \in L\Gamma$.
					By the definition of $L\Gamma$, there is a net $(x_i)_{i \in I}$
					so that $x_i \in \text{span}\set{\lambda_g : g \in \Gamma}$
					for each $i \in I$ and $x_i \to x$ SOT.
					By the Kaplansky density theorem,
					we may assume $\|x_i\| \le \|x\|$ for all $i \in I$.
					Then since multiplication is jointly continuous on bounded sets,
					$(x - x_i)^*(x - x_i) \to 0$ SOT.
					Therefore, $(x - x_i)^*(x - x_i) \to 0$ WOT and hence $\sigma$-WOT,
					since everything is bounded.
					Now since $\tr$ is $\sigma$-WOT continuous,
					$$\|x\Omega - x_i\Omega\|_2^2 = \tr((x - x_i)^*(x - x_i)) \to \tr(0) = 0.$$
					That is, $x_i\Omega \to x\Omega$ in $L^2L\Gamma$.
					This completes the proof that $(\lambda_g)_{g \in \Gamma}$
					is an orthonormal basis for $L^2L\Gamma$.
					
					Now for all $g, h \in \Gamma$,
					\begin{align*}
						L_{\lambda_g}u\delta_h
						& = L_{\lambda_g}\lambda_h\Omega \\
						& = \lambda_g\lambda_h\Omega \\
						& = \lambda_{gh}\Omega \\
						& = u\delta_{gh} \\
						& = u\lambda_g\delta_h.
					\end{align*}
					Hence by linearity and continuity of $u$ and each $\lambda_g$, we have
					$L_xu\xi = ux\xi$ for all $\xi \in \ell^2\Gamma$ and all
					$x \in \text{span}\set{\lambda_g : g \in \Gamma}$.
					It remains to check that if $x_i \to x$ SOT, then $L_{x_i} \to L_x$ SOT.
					By the Kaplansky density theorem, we may assume $\|x_i\| \le \|x\|$ for all $i$.
					Then for $y \in L\Gamma$, we have $\|x_iy\| \le \|xy\|$ and $x_iy \to xy$ SOT.
					Hence, by the computation in the previous paragraph,
					$L_{x_i}(y\Omega) \to L_x(y\Omega)$ in $L^2L\Gamma$.
					Since this holds for all $y \in L\Gamma$, we have $L_{x_i} \to L_x$ SOT
					as desired.
				\end{proof}
			\item Deduce from Problem 91 that $L\Gamma' = R\Gamma$.
				\begin{proof}
					By problem 91, it suffices to show $J L\Gamma J = R\Gamma$.
					For $g, h \in \Gamma$, we have
					\begin{align*}
						(J L_x J) (\lambda_h\Omega)
						 & = J L_x (\lambda_{h^{-1}}\Omega) \\
						 & = J (x\lambda_{h^{-1}} \Omega) \\
						 & = \lambda_hx^* \Omega \\
						 & = R_{x^*}(\lambda_h \Omega).
					\end{align*}
					Since $(\lambda_h\Omega)_{h \in \Gamma}$ is an orthonormal basis for
					$L^2L\Gamma$, we have $JL_xJ = R_{x^*}$.
					Therefore, $J L\Gamma J = R\Gamma$.
				\end{proof}
		\end{enumerate}
\end{enumerate}

\end{document}          
