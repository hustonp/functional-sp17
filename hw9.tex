\documentclass[a4paper,10pt]{report}
\usepackage[utf8]{inputenc}
\usepackage{amsmath}
\usepackage{amssymb}
\usepackage{amsthm}
\usepackage{mathtools}
\usepackage{fancyhdr}
\usepackage{enumitem}
\usepackage[top=1in,left=1in,right=1in]{geometry}
\usepackage{mathrsfs}
\usepackage{bm}

\usepackage{bbm}
\usepackage{tikz-cd}
\usepackage{stackengine}
\usepackage{Math_Symbols} 
\usepackage{ifpdf}
\ifpdf
%\usepackage[pdftex]{graphicx}
%\else
%\usepackage[dvips]{graphicx}
%\fi

\setenumerate{listparindent=\parindent,topsep=\parskip}
% \setlist[enumerate]{topsep=\parskip}
\setlist[enumerate,2]{label=(\arabic*)}
\setlist[enumerate,3]{label=(\alph*)}

% \newcommand{\set}[1]{{\{#1\}}}
\newcommand{\ggen}[1]{\langle#1\rangle}
\newcommand{\pn}[2]{||#1||_{#2}}
\newcommand{\bpn}[2]{\left|\left|#1\right|\right|_{#2}}
\newcommand{\norm}[1]{||#1||}
\newcommand{\bnorm}[1]{\left|\left|#1\right|\right|}
\DeclarePairedDelimiter{\ceil}{\lceil}{\rceil}
\DeclarePairedDelimiter{\floor}{\lfloor}{\rfloor}
\DeclarePairedDelimiter{\set}{\{}{\}}
\DeclarePairedDelimiter{\abs}{|}{|}
\DeclarePairedDelimiter{\ket}{|}{\rangle}
\DeclarePairedDelimiter{\bra}{\langle}{|}

\newcommand{\ol}[1]{\overline{#1}}

\renewcommand{\mod}{\text{ mod }}

\renewcommand{\O}{\operatorname{O}} % Bound otherwise
\renewcommand{\o}{\operatorname{o}}
\newcommand{\T}{\text{yes}}
\newcommand{\F}{\text{no}}

\newcommand{\Z}{\mathbb{Z}}
\newcommand{\N}{\mathbb{N}}
\newcommand{\C}{\mathbb{C}}
\newcommand{\Q}{\mathbb{Q}}
\newcommand{\textns}{\otimes}
\newcommand{\rar}[2][]{\overset{#2}{\underset{#1}{\longrightarrow}}}

\DeclareMathOperator{\img}{img}
\DeclareMathOperator{\fop}{int}
\DeclareMathOperator{\fcl}{cl}
% \DeclareMathOperator{\lg}{lg}
\DeclareMathOperator{\vspan}{span}
\DeclareMathOperator{\rng}{rng}
\DeclareMathOperator{\Rng}{Rng}
\DeclareMathOperator{\Cov}{Cov}
\DeclareMathOperator{\Var}{Var}
\DeclareMathOperator{\Bernoulli}{Bernoulli}
\DeclareMathOperator{\Normal}{Normal}
\DeclareMathOperator{\Uniform}{Uniform}
\DeclareMathOperator{\Binom}{Binomial}
\DeclareMathOperator{\mgf}{mgf}
\DeclareMathOperator{\Supp}{Supp}

\newcommand{\cat}[1]{(\bm{#1})}

\providecommand{\Alpha}{A}

\newtheorem*{lemma*}{Lemma}

% Roman numerals
\makeatletter
\newcommand{\Romnum}[1]{\expandafter\@slowromancap\romannumeral #1@}
\makeatother
\newcommand{\factor}[1]{\text{\Romnum{#1}}}

\begin{document}
% \maketitle

\pagestyle{fancy}	
\fancyhf{} % Reset headers and footers
\lhead{Ethan Ackelsberg, Zachery Dell, Peter Huston\\
Functional Analysis 2\\
\today}
\setlength{\headheight}{60pt}

\begin{center}
	\textbf{Homework 9}
\end{center}

\begin{enumerate}
		\setcounter{enumi}{92}
	\item 
		\begin{enumerate}
			\item Since convolution is $\C$-linear and associative, $\ell\Gamma$ is a $\C$-module, closed under convolution. We know that $\delta_e\in\ell^2\Gamma$ is a unit for the convolution, with $\delta_e\ast f$ defined for every $f:\Gamma\to\C$. Therefore, $\delta_e\in\ell\Gamma$, making $\ell\Gamma$ a unital algebra. We will use part (2) to prove that $\ell\Gamma$ is $*$-closed. (Contrary to the hint, we will not use $*$-closure of $\ell\Gamma$ in proving (2)). 

				Let $x\in\ell\Gamma$. First, notice that the definition of $x^*$ by $(x^*)_g=\ol{x_{g^{-1}}}$ is the only possible definition so that $(T_x)^*=T_{x^*}$, because if $T_x$ has an adjoint at all, we must have $\ggen{T_x\delta_e,\delta_e}=\ggen{\delta_e,T_x^*\delta_e}$. Moreover, for every $x,\eta\in\ell^2\Gamma$, by {H\"{o}lder's} inequality, $x\ast\eta\in\ell^\infty\Gamma$ is well-defined. The only issue is checking that $\pn{x\ast\eta}{2}<\infty$. Suppose $\xi\in(\ell^2\Gamma)_1$. % with $\pn{x^*\ast\xi}{2}>\pn{x}^2$. % Weird thought, but leaving as hint because I doubt the whole argument. 
				It would be circular to assert that $\ggen{x\ast x^*\ast\xi,\xi}=\pn{x^*\ast\xi}{2}$, but we can approximate. For $F\subseteq\Gamma$ finite, define $\eta_F=\sum_{g\in F}(x^*\ast\xi)_g\delta_g$. Then we have $\ggen{\eta_F,x^*\ast\xi}=\pn{\eta_F}{2}^2\uparrow\pn{x^*\ast\xi}{2}^2$. Since each $\eta_F$ is in $\ell^1\Gamma$, we can apply Fubini's theorem, obtaining 
				\begin{align*}
					\ggen{\eta_F,x^*\ast\xi} &= \ggen{x\ast\eta_F,\xi}\\
					&\le \norm{T_x}\cdot\norm{\eta_F}{2}\norm{\xi}{2}\\
					&\le \norm{T_x}\cdot\pn{\xi}{2}^2\\
					\intertext{Taking limits, }
					\pn{x^*\ast\xi}{2}^2 &\le \norm{T_x}\cdot\pn{\xi}{2}^2
				\end{align*}
				so $\norm{x^*}\le\norm{x}$ as usual. In particular, $\pn{x^*\ast\xi}{2}<\infty$, showing that $x^*\in\ell\Gamma$. 
			\item Suppose $x\in\ell\Gamma$. To show that $x\in B(\ell^2\Gamma)$, we apply the closed graph theorem. It suffices to check that if $\eta_n\to\eta$ in $\ell^2\Gamma$ and $x\ast\eta_n\to\xi$ in $\ell^2\Gamma$, then $x\ast\eta=\xi$. The following calculation is basic, but in the interest of full disclosure, I read it in Jesse's notes while looking for something else. If $g\in\Gamma$, then by continuity of the inner product, 
				\begin{align*}
					|(\xi-x\ast\eta)_g| &= \lim_{n\to\infty}|(x\ast\eta_n-x\ast\eta)_g|\\
					&\le \lim_{n\to\infty}\pn{(x\ast(\eta_n-\eta))_g}{\infty}\\
					&\le \lim_{n\to\infty}\pn{x}{2}\pn{\eta_n-\eta}{2}\\
					&= 0
				\end{align*}
				Therefore, $\xi$ and $x\ast\eta$ agree pointwise, and hence $\xi=x\ast\eta$, as desired. 
			\item Since $L\Gamma=R\Gamma'=\set{\rho_g:g\in\Gamma}'''=\set{\rho_g:g\in\Gamma}'$ as a subalgebra of $B(\ell^2\Gamma)$, it suffices to check that for every $x\in\ell\Gamma$ and $g\in\Gamma$, we have $x\rho_g=\rho_gx$. Let $x\in\ell\Gamma$ and $g\in\Gamma$ be given. Since $x$ and $\rho_g$ are norm-continuous, it suffices to check that for every $h\in\Gamma$, $x\ast\rho_g(\delta_h)=\rho_g(x\ast\delta_h)$. Let $h\in\Gamma$ be given. Then we can compute that 
				\begin{align*}
					x\ast\rho_g(\delta_h) &= x\ast\delta_{hg^{-1}}\\
					\intertext{ and for every $k\in\Gamma$, }
					(x\ast\rho_g(\delta_h))_k &= (x\ast\delta_{hg^{-1}})_k\\
					&= x_{kgh^{-1}}\\
					&= (x\ast\delta_h)_{kg}\\
					&= \rho_g(x\ast\delta_h)_k
					\intertext{ showing that }
					x\ast\rho_g(\delta_h) &= \rho_g(x\ast\delta_h)\\
					\intertext{ showing that }
					T_x\rho_g &= \rho_gT_x
				\end{align*}
				as desired. 
			\item It is clear that $x\to T_x$ is a unital homomorphism. We saw when solving part (1) that $(T_x)^*=T_{x^*}$. Because $\ggen{T_x\delta_e,\delta_g}=x_g$, an inverse homomorphism is given by $T\to(\ggen{T\delta_e,\delta_g})_{g\in\Gamma}$; that the range of this homorphism is contained in $\ell\Gamma$ was proven in class. 
		\end{enumerate}
		\setcounter{enumi}{94}
	      \item Let $M$ and $N$ be von Neumann algebras, and $\Ed : M \to N$ a *-isomorphism.\\
                First, $\Ed$ sends positive operators to positive operators, since $\Ed \E x^* x \R = \Ed \E x \R^* \Ed \E x \R$.\\
                Let $\E x_\lambda \R$ be an increasing net of positive operators in $M$, with least upper bound $x$.\\
                Lastly, if $y \in N$ with $\Ed \E x \R \geq y \geq \Ed \E x_\lambda \R$ for all $\lambda$, then $x \geq \Ed^{-1} \E y \R \geq x_\lambda$ for all $\lambda$, and thus $y = x$, because $x$ is the least upper bound for $\E x_\lambda \R$.\\
                Thus $\Ed \E x_\lambda \R \toup \Ed \E x \R$, so $\Ed$ is normal.\\
                $\qed$







	\item 
		\begin{enumerate}
			\item A well-known result in group theory says that any subgroup of a free group is free. Therefore, it suffices to prove that the conjugacy classes of the members of a generating set for $\mathbb{F}_2$ are infinite. We know that in $S_n$, we have $(1,2,\ldots n)^{k}(1,2)(1,2\ldots n)^{-k}=(k+1,k+2)$. Therefore, for every $n\in\N$, there exist groups $G$ with elements $a,b\in G$ such that $a^kba^{-k}$ are distinct for every $k\le n$. If $\mathbb{F}_2=\ggen{x,y}$, then $|\set{x^kyx^{-k}|}$ is either infinite, or a non-zero integer multiple of every natural number. Therefore, $y$ has infinite conjugacy class, as desired. 
				
				Since $\mathbb{F}_2$ is an ICC group, by a result proved in class, $L\mathbb{F}_2$ is a $\factor{2}_1$ factor. 
			\item Let $\Phi:\Gamma\to\Lambda$ be an isomorphism of groups. Then there is an isometric isomorphism $\Phi_*:\ell^2\Lambda\to\ell^2\Gamma$, defined by $\Phi_*(\eta)_g=\eta_{\Phi(g)}$ for $g\in\Gamma$, with $(\Phi^{-1})_*=(\Phi_*)^{-1}$. Pick $x\in\ell\Lambda$. For every $\eta\in\ell^2\Gamma$, we have $\Phi_*(x)\ast\eta=\Phi_*(x\ast\Phi_*^{-1}(\eta))$, so $\Phi_*(x)\in\ell\Gamma$. Therefore, we may consider $\Phi_*(X)$ as an algebra isomorphism $\ell\Lambda\to\ell\Gamma$. Since $\Phi$ preserves the group identity, $\Phi_*(\delta_e)=\delta_e$, showing that $\Phi_*$ is unital, and since $\Phi$ preserves inverses and is linear, $\Phi_*$ is a $*$-homomorphism. By problem 93 part (4), we can turn $\Phi_*$ into a unital $*$-isomorphism $L\Lambda\to L\Gamma$. 

				In particular, $\sigma$ is an automorphism of $\mathbb{F}_2$, so $\sigma$ defines an automorphism $\alpha$ of $L\mathbb{F}_2$. 
			\item We will show that $\alpha$ is free, which implies that $\alpha$ is outer. Happily for us, $\sigma=\sigma^{-1}$. Therefore, we can simply write, for $x\in\ell\mathbb{F}_2$, that $\alpha(x)_g=x_{\sigma(g)}$. Now for every $g,h\in\mathbb{F}_2$ and $x\in\ell\mathbb{F}_2$, we have $(x\ast\delta_g)_h=x_{h^{-1}g}$ and $(\delta_g\ast x)_h=x_{g^{-1}h}$. Suppose $x\in\ell\mathbb{F}_2$ such that for every $g\in\mathbb{F}_2$, we have $x\ast\alpha(\delta_{g^{-1}})=\delta_{g^{-1}}\ast x$; then for every $g,h\in\mathbb{F}_2$, we get $x_{h\sigma(g)}=x_{gh}$. In particular, if $\mathbb{F}_2\cong\ggen{a,b}$ is a presentation, then for every $w\in\mathbb{F}_2$, we have $x_w=x_{a(a^{-1}w)}=x_{a^{-1}wb}$. Of course, counting up $a$'s and $b$'s via the homormohpism $\ggen{a,b}\to\ggen{a,b|[a,b]}\cong\Z^2$ shows that for any $w$, the set $\set{a^{-n}wb^n:n\in\Z}$ is infinite. This shows that $x$ is constant on infinite subsets which cover $\mathbb{F}_2$. Since $x\in\ell^2\mathbb{F}_2$, this means that $x=0$. By definition, we have shown that $\alpha$ is free. 
		\end{enumerate}











              \item Let $\ma Z$ act on $L^\infty \E S^1 \R$ via irrational rotation, i.e., let $T : \ma T \to \ma T$ be given by $T \E x \R = x + \A$ for some $\A \in \ma R \back \ma Q$, and define the action by $n \mapsto \blank \circ T^n$.\\
                First we show that the action is ergodic.  Let $S \subs \ma T$ be measurable and invariant under $T$.\\
                Fix $\ep > 0$.  Since $C \E \ma T \R$ is dense in $L^1 \E \ma T \R$, we can find $f \in C \E \ma T \R$ such that

                \begin{align*}
                  \NK f - \mb 1_S \KN_1 < \fa{\ep}{2}.
                \end{align*}

                Then since $S$ is $T$ invariant, we also have that for each $n \in \ma Z$,

                \begin{align*}
                  \NK f \circ T^n - \mb 1_S \KN_1 < \fa{\ep}{2}.
                \end{align*}

                Thus for each $n \in \ma Z$, $\NK f \circ T^n - f \KN_1 < \ep$.\\
                But since $x + n \A$ is dense in $\ma T$, it follows that for each $t \in \ma T$, $\NK f \E x + t \R - f \E x \R \KN_1 < \ep$, bceause $f$ is continuous.\\
                We claim that $f$ is constant: ($\tau_x$ is translation by $x$):

                \begin{align*}
                  \NK f - \I f \circ \tau_x \E t \R dt \KN_1%
                  &= \I \MG f - \I f \circ \E x + t \R dt \GM dx\\
                  &\leq \MG \I \I f \E x \R - f \E x + t \R dt dx \GM\\
                  &= \MG \I \I f \E x \R - f \E x + t \R dx dt \GM , \quad \te{ by Fubini}\\
                  &= \I \NK f - f \circ \tau_x \KN_1 dt\\
                  &= \NK f - f \circ \tau_x \KN_1\\
                  &\leq \ep.
                \end{align*}

                Thus $f$ is equal a.e. to it's average value, and thus is constant.  It follows that $\mu \E S \R = 0$ or $\mu \E S \R = 1$, so the action is ergodic.\\


                Now we show that the action is free.\\
                







\end{enumerate}

\end{document}          
