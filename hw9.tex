\documentclass[a4paper,10pt]{report}
\usepackage[utf8]{inputenc}
\usepackage{amsmath}
\usepackage{amssymb}
\usepackage{amsthm}
\usepackage{mathtools}
\usepackage{fancyhdr}
\usepackage{enumitem}
\usepackage[top=1in,left=1in,right=1in]{geometry}
\usepackage{mathrsfs}
\usepackage{bm}

\usepackage{bbm}
\usepackage{tikz-cd}
\usepackage{stackengine}
\usepackage{Math_Symbols} 
\usepackage{ifpdf}
\ifpdf
%\usepackage[pdftex]{graphicx}
%\else
%\usepackage[dvips]{graphicx}
%\fi

\setenumerate{listparindent=\parindent,topsep=\parskip}
% \setlist[enumerate]{topsep=\parskip}
\setlist[enumerate,2]{label=(\arabic*)}
\setlist[enumerate,3]{label=(\alph*)}

% \newcommand{\set}[1]{{\{#1\}}}
\newcommand{\ggen}[1]{\langle#1\rangle}
\newcommand{\pn}[2]{||#1||_{#2}}
\newcommand{\bpn}[2]{\left|\left|#1\right|\right|_{#2}}
\newcommand{\norm}[1]{||#1||}
\newcommand{\bnorm}[1]{\left|\left|#1\right|\right|}
\DeclarePairedDelimiter{\ceil}{\lceil}{\rceil}
\DeclarePairedDelimiter{\floor}{\lfloor}{\rfloor}
\DeclarePairedDelimiter{\set}{\{}{\}}
\DeclarePairedDelimiter{\abs}{|}{|}
\DeclarePairedDelimiter{\ket}{|}{\rangle}
\DeclarePairedDelimiter{\bra}{\langle}{|}

\newcommand{\ol}[1]{\overline{#1}}

\renewcommand{\mod}{\text{ mod }}

\renewcommand{\O}{\operatorname{O}} % Bound otherwise
\renewcommand{\o}{\operatorname{o}}
\newcommand{\T}{\text{yes}}
\newcommand{\F}{\text{no}}

\newcommand{\Z}{\mathbb{Z}}
\newcommand{\N}{\mathbb{N}}
\newcommand{\C}{\mathbb{C}}
\newcommand{\Q}{\mathbb{Q}}
\newcommand{\textns}{\otimes}
\newcommand{\rar}[2][]{\overset{#2}{\underset{#1}{\longrightarrow}}}

\DeclareMathOperator{\img}{img}
\DeclareMathOperator{\fop}{int}
\DeclareMathOperator{\fcl}{cl}
% \DeclareMathOperator{\lg}{lg}
\DeclareMathOperator{\vspan}{span}
\DeclareMathOperator{\rng}{rng}
\DeclareMathOperator{\Rng}{Rng}
\DeclareMathOperator{\Cov}{Cov}
\DeclareMathOperator{\Var}{Var}
\DeclareMathOperator{\Bernoulli}{Bernoulli}
\DeclareMathOperator{\Normal}{Normal}
\DeclareMathOperator{\Uniform}{Uniform}
\DeclareMathOperator{\Binom}{Binomial}
\DeclareMathOperator{\mgf}{mgf}
\DeclareMathOperator{\Supp}{Supp}

\newcommand{\cat}[1]{(\bm{#1})}

\providecommand{\Alpha}{A}

\newtheorem*{lemma*}{Lemma}

% Roman numerals
\makeatletter
\newcommand{\Romnum}[1]{\expandafter\@slowromancap\romannumeral #1@}
\makeatother
\newcommand{\factor}[1]{\text{\Romnum{#1}}}

\begin{document}
% \maketitle

\pagestyle{fancy}	
\fancyhf{} % Reset headers and footers
\lhead{Ethan Ackelsberg, Zachery Dell, Peter Huston\\
Functional Analysis 2\\
\today}
\setlength{\headheight}{60pt}

\begin{center}
	\textbf{Homework 9}
\end{center}

\begin{enumerate}
		\setcounter{enumi}{92}
	\item 
		\begin{enumerate}
			\item Since convolution is $\C$-linear and associative, $\ell\Gamma$ is a $\C$-module, closed under convolution. We know that $\delta_e\in\ell^2\Gamma$ is a unit for the convolution, with $\delta_e\ast f$ defined for every $f:\Gamma\to\C$. Therefore, $\delta_e\in\ell\Gamma$, making $\ell\Gamma$ a unital algebra. We will use part (2) to prove that $\ell\Gamma$ is $*$-closed. (Contrary to the hint, we will not use $*$-closure of $\ell\Gamma$ in proving (2)). 

				Let $x\in\ell\Gamma$. First, notice that the definition of $x^*$ by $(x^*)_g=\ol{x_{g^{-1}}}$ is the only possible definition so that $(T_x)^*=T_{x^*}$, because if $T_x$ has an adjoint at all, we must have $\ggen{T_x\delta_e,\delta_e}=\ggen{\delta_e,T_x^*\delta_e}$. Moreover, for every $x,\eta\in\ell^2\Gamma$, by {H\"{o}lder's} inequality, $x\ast\eta\in\ell^\infty\Gamma$ is well-defined. The only issue is checking that $\pn{x\ast\eta}{2}<\infty$. Suppose $\xi\in(\ell^2\Gamma)_1$. % with $\pn{x^*\ast\xi}{2}>\pn{x}^2$. % Weird thought, but leaving as hint because I doubt the whole argument. 
				It would be circular to assert that $\ggen{x\ast x^*\ast\xi,\xi}=\pn{x^*\ast\xi}{2}$, but we can approximate. For $F\subseteq\Gamma$ finite, define $\eta_F=\sum_{g\in F}(x^*\ast\xi)_g\delta_g$. Then we have $\ggen{\eta_F,x^*\ast\xi}=\pn{\eta_F}{2}^2\uparrow\pn{x^*\ast\xi}{2}^2$. Since each $\eta_F$ is in $\ell^1\Gamma$, we can apply Fubini's theorem, obtaining 
				\begin{align*}
					\ggen{\eta_F,x^*\ast\xi} &= \ggen{x\ast\eta_F,\xi}\\
					&\le \norm{T_x}\cdot\norm{\eta_F}{2}\norm{\xi}{2}\\
					&\le \norm{T_x}\cdot\pn{\xi}{2}^2\\
					\intertext{Taking limits, }
					\pn{x^*\ast\xi}{2}^2 &\le \norm{T_x}\cdot\pn{\xi}{2}^2
				\end{align*}
				so $\norm{x^*}\le\norm{x}$ as usual. In particular, $\pn{x^*\ast\xi}{2}<\infty$, showing that $x^*\in\ell\Gamma$. 
			\item Suppose $x\in\ell\Gamma$. To show that $x\in B(\ell^2\Gamma)$, we apply the closed graph theorem. It suffices to check that if $\eta_n\to\eta$ in $\ell^2\Gamma$ and $x\ast\eta_n\to\xi$ in $\ell^2\Gamma$, then $x\ast\eta=\xi$. The following calculation is basic, but in the interest of full disclosure, I read it in Jesse's notes while looking for something else. If $g\in\Gamma$, then by continuity of the inner product, 
				\begin{align*}
					|(\xi-x\ast\eta)_g| &= \lim_{n\to\infty}|(x\ast\eta_n-x\ast\eta)_g|\\
					&\le \lim_{n\to\infty}\pn{(x\ast(\eta_n-\eta))_g}{\infty}\\
					&\le \lim_{n\to\infty}\pn{x}{2}\pn{\eta_n-\eta}{2}\\
					&= 0
				\end{align*}
				Therefore, $\xi$ and $x\ast\eta$ agree pointwise, and hence $\xi=x\ast\eta$, as desired. 
			\item 
			\item 
		\end{enumerate}
		\setcounter{enumi}{94}
	\item 
	\item 
		\begin{enumerate}
			\item 
			\item 
			\item 
		\end{enumerate}
	\item 
\end{enumerate}

\end{document}          
