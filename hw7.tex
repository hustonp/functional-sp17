\documentclass[a4paper,10pt]{report}
\usepackage[utf8]{inputenc}
\usepackage{amsmath}
\usepackage{amssymb}
\usepackage{amsthm}
\usepackage{mathtools}
\usepackage{fancyhdr}
\usepackage{enumitem}
\usepackage[top=1in,left=1in,right=1in]{geometry}
\usepackage{mathrsfs}
\usepackage{bm}

% Blind inclusion of Zach's header.
\usepackage{bbm}
% \usepackage{mathtools}
\usepackage{tikz-cd}
\usepackage{stackengine}
% \usepackage{Math_Symbols} % couldn't find
% \usepackage{amsmath, amsthm, amssymb}
% \usepackage{mathrsfs}
\usepackage{ifpdf}
\ifpdf
%\usepackage[pdftex]{graphicx}
%\else
%\usepackage[dvips]{graphicx}
%\fi

\setenumerate{listparindent=\parindent,topsep=\parskip}
% \setlist[enumerate]{topsep=\parskip}
\setlist[enumerate,2]{label=(\arabic*)}
\setlist[enumerate,3]{label=(\alph*)}

% \newcommand{\set}[1]{{\{#1\}}}
\newcommand{\ggen}[1]{\langle#1\rangle}
\newcommand{\pn}[2]{||#1||_{#2}}
\newcommand{\bpn}[2]{\left|\left|#1\right|\right|_{#2}}
\newcommand{\norm}[1]{||#1||}
\newcommand{\bnorm}[1]{\left|\left|#1\right|\right|}
\DeclarePairedDelimiter{\ceil}{\lceil}{\rceil}
\DeclarePairedDelimiter{\floor}{\lfloor}{\rfloor}
\DeclarePairedDelimiter{\set}{\{}{\}}
\DeclarePairedDelimiter{\abs}{|}{|}
\DeclarePairedDelimiter{\ket}{|}{\rangle}
\DeclarePairedDelimiter{\bra}{\langle}{|}

\newcommand{\op}[1]{\mathring{#1}}
\newcommand{\cl}[1]{\overline{#1}}
\newcommand{\ol}[1]{\overline{#1}}

\renewcommand{\mod}{\text{ mod }}

\renewcommand{\O}{\operatorname{O}} % Bound otherwise
\renewcommand{\o}{\operatorname{o}}
\newcommand{\T}{\text{yes}}
\newcommand{\F}{\text{no}}

\newcommand{\Z}{\mathbb{Z}}
\newcommand{\R}{\mathbb{R}}
\newcommand{\N}{\mathbb{N}}
\newcommand{\C}{\mathbb{C}}
\newcommand{\Q}{\mathbb{Q}}
\newcommand{\Prob}{\mathbb{P}}
\newcommand{\E}{\mathbb{E}}
\newcommand{\Ba}{\ensuremath Ba}
\newcommand{\del}{\partial}
\newcommand{\textns}{\otimes}
\newcommand{\rar}[2][]{\overset{#2}{\underset{#1}{\longrightarrow}}}

\DeclareMathOperator{\img}{img}
\DeclareMathOperator{\sgn}{sgn}
\DeclareMathOperator{\fop}{int}
\DeclareMathOperator{\fcl}{cl}
% \DeclareMathOperator{\lg}{lg}
\DeclareMathOperator{\vspan}{span}
\DeclareMathOperator{\diam}{diam}
\DeclareMathOperator{\id}{id}
% \DeclareMathOperator{\Im}{Im}
% \DeclareMathOperator{\Rp}{Rp}
\DeclareMathOperator{\rng}{rng}
\DeclareMathOperator{\Spec}{spec}
\DeclareMathOperator{\spec}{spec} % match Zach
\DeclareMathOperator{\Rng}{Rng}
\DeclareMathOperator{\Cov}{Cov}
\DeclareMathOperator{\Var}{Var}
\DeclareMathOperator{\Bernoulli}{Bernoulli}
\DeclareMathOperator{\Normal}{Normal}
\DeclareMathOperator{\Uniform}{Uniform}
\DeclareMathOperator{\Binom}{Binomial}
\DeclareMathOperator{\mgf}{mgf}
\DeclareMathOperator{\Supp}{Supp}
\DeclareMathOperator{\Tr}{Tr}
\DeclareMathOperator{\tr}{tr}

\newcommand{\cat}[1]{(\bm{#1})}

\providecommand{\Alpha}{A}

\newtheorem*{lemma*}{Lemma}

% Roman numerals
\makeatletter
\newcommand{\Romnum}[1]{\expandafter\@slowromancap\romannumeral #1@}
\makeatother
\newcommand{\factor}[1]{\text{\Romnum{#1}}}

\begin{document}
% \maketitle

\pagestyle{fancy}	
\fancyhf{} % Reset headers and footers
\lhead{Ethan Ackelsberg, Zachery Dell, Peter Huston\\
Functional Analysis 2\\
\today}
\setlength{\headheight}{60pt}

\begin{center}
	\textbf{Homework 7}
\end{center}

\begin{enumerate}
		\setcounter{enumi}{83}
 \item 
 
 Let $S_{\infty}$ be the group of finite permutations of $\N$.
		\begin{enumerate}
			\item	Show that $S_{\infty}$ is ICC. Deduce that $LS_{\infty}$ is a $\factor{2}_1$ factor.
			\begin{proof}
			Fix $n\ge 2$. We may naturally identify $S_n\cong\set{\phi\in S_{\infty}:\phi(m)=m\text{ for all }m>n}\subseteq S_\infty$. There are infinitely many subgroups conjugate to $S_n$ in $S_\infty$ with trivial intersection: if $\phi_{n,k}=\prod_{j=1}^n(j,kn+j)$, then $\phi_{n,k}S_n\phi_{n,k}\bigcap\phi_{n,i}S_n\phi_{n,i}$ is trivial whenever $i\neq j$. Since every $\phi\in S_\infty$ lies in some $S_n$, if $\phi$ is not the identity, there are infinitely many distinct conjugates of $\phi$ in $S_\infty$. Therefore, $S_\infty$ is ICC. We proved in class that this makes $LS_\infty$ a type $\factor{2}_1$ factor. 
			\end{proof}
			\item	Give an explicit description of a projection with trace $k^{-n}$ for arbitrary $n, k \in \N$.
			\begin{proof}
			Let an integer $m=k^n$ be given. Let $H$ be a subgroup of $S_\infty$ of order $m$, for example generated by a cycle of length $m$. Set $p=m^{-1}\sum_{g\in H}\lambda_g$. Since $\lambda_{g^{-1}}=\lambda_g^*$, this $p$ is self-adjoint. Also, $p^2=m^{-2}\sum_{(g,h)\in H\times H}\lambda_g\lambda_h=m^{-1}\sum_{k\in H}\lambda_k=p$. Therefore, $p\in P(L\Gamma)$. Of course, $\tr(p)=\ggen{p\delta_e,\delta_e}=m^{-1}\sum_{g\in H}\ggen{\delta_g,\delta_e}=m^{-1}$, as desired. 
			\end{proof}
			\item	Find an increasing sequence $F_n \subseteq LS_{\infty}$ of finite dimensional von Neumann subalgebras such that $LS_{\infty} = \left( \bigcup_{n=1}^{\infty}{F_n} \right)$.
			\begin{proof}
			Since $S_\infty=\bigcup_{n\in\N}S_n$, we have $\C S_\infty=\bigcup_{n\in\N}\C S_n$, and $LS_\infty=(\C S_\infty)''$ by definition. 
			\end{proof}
		\end{enumerate}
		\setcounter{enumi}{85}

\item
 Let $M$ be a factor.
 Prove that if $M$ is finite or purely infinite, then $M$ is algebraically simple.
 \begin{proof}
 	We will deal with the three cases separately:
	type $\factor{1}_n$ for $n \in \N$, type $\factor{2}_1$, and type \factor{3}.
	First suppose $M$ is type $\factor{1}_n$.
	Then $M \cong M_n(\C)$.
	Suppose $I \subseteq M_n(\C)$ is a nonzero two-sided ideal.
	Let $A \in I$ be a nonzero matrix.
	Then taking $\left(e_i\right)_{i=1}^n$ as the standard basis vectors,
	we have $\alpha = \bra{e_i}A\ket{e_j} \ne 0$ for some $1 \le i, j \le n$.
	Now observe that
	$$p_k = \ket{e_k} \bra{e_k} = \frac{1}{\alpha}\ket{e_k}\bra{e_i} A \ket{e_j} \bra{e_k} \in I$$
	for $1 \le k \le n$.
	Summing over $k$, we have $1 = \sum_{k=1}^n{p_k} \in I$.
	
	Now suppose $M$ is type $\factor{2}_1$.
	Let $\tr$ be a $\sigma$-WOT continuous (faithful) tracial state on $M$.
	Let $I \subseteq M$ be a nonzero two-sided ideal.
	By the Borel functional calculus, $I$ contains a nonzero projection $p$.
	We claim that for any projection $q \in P(M)$, if $\tr(q) \le \tr(p)$, then $q \in I$.
	Let $q$ be such a projection.
	Then $q \preceq p$, so there is a partial isometry $u \in M$
	such that $uu^* = q$ and $u^*u \le p$.
	Since $I$ is a two-sided ideal, we have
	$$q = uu^* = u(u^*u)u^* = u(u^*up)u^* \in I.$$
	This proves the claim that $I \supseteq \{q \in P(M) : \tr(q) \le \tr(p)\}$.
	If $\tr(p) \ge \frac{1}{2}$, this implies immediately that $1 - p \in I$, whence $1 = p + (1 - p) \in I$.
	Suppose $\tr(p) < \frac{1}{2}$.
	Then $p \preceq 1 - p$, so there is a projection $q \le 1 - p$ with $q \approx p$.
	By the above claim, $q \in I$, so $p + q \in I$ and $\tr(p + q) = 2\tr(p)$.
	Thus, we may find a projection in $I$ with trace double that of $p$.
	Repeating this if necessary, we eventually arrive at a projection of trace at least $\frac{1}{2}$,
	so by the previous case, $I = M$.

		Finally, consider the case that $M$ is a purely infinite factor, that is, type $\factor{3}$, and suppose that $M$ is countably decomposable. In this case, we adapt a proof from Jesse Peterson's notes on Von Neumann algebras. Let $p\in P(M)\setminus\set{0}$ be a non-zero projection. We will show that $p\approx 1$. First, we find $q\le p$ with $q\approx 1-q\approx p$. Since $p$ must be infinite, there is some partial isometry $u\in M$ with $u^*u=p$ and $uu^*<p$ yet $uu^*\approx p$. Define $p_0=1-uu^*$, and for $n\in\N$, define $p_n=u^np_0u^*$. Then $p_n$ is a projection, and since $uu^*\le p$ and $u^*u\le p$, we have $p_n\le p$. 
		
		For $n>m$, we have 
		\begin{align*}
			p_np_m &= (u^n(1-uu^*)(u^*)^n)(u^m(1-uu^*)(u^*)^m)\\
			&= u^n(u^*)^nu^m(u^*)^m-u^n(u^*)^nu^{m+1}(u^*)^{m+1}-u^{n+1}(u^*)^{n+1}u^m(u^*)^m+u^{n+1}(u^*)^{n+1}u^{m+1}(u^*)^{m+1}\\
			\intertext{Since $uu^*\le p$ and $u^*u\le p$, the presence of $p=u^*u$ in the above terms is redundant. Cancelling $m+1$ times gives }
			p_np_m &= u^n(u^*)^{n-m}u^*m-u^n(u^*){n-(m+1)}(u^*)^{m+1}-u^{n+1}(u^*)^{(n+1)-m}(u^*)^m+u^{n+1}(u^*)^{n-m}(u^*)^{m+1}\\
			&= 0
		\end{align*}
		Also notice that just as $up_nu^*=p_{n+1}$, we have $u^*p_nu=(u^*u)u^{n-1}p_0(u^*)^{n-1}(u^*u)=p_{n-1}$; for all $n$ and $m$, we have $p_n\approx p_m$. Therefore, $(p_n)_{n\in\N}$ is a sequence of mutually orthogonal projections, each equivalent to $p_0$ and bounded above by $p$. 

		By Zorn's lemma, we may extend $(p_n)_{n\in\omega}$ to a maximal family of mutually orthogonal equivalent projections bounded above by $p$, say $(q_n)_{n\in\omega}$; we may assume the family is countable since $M$ is countably decomposable. Let $q_\omega=1-\sum_{n\in\omega}q_n$. If $q_\omega\succ q_0$, then there exists $v$ a partial isometry with $v^*v\le q_\omega$ and $vv^*\approx p_0$, contradicting the maximality of $(q_n)_{n\in\omega}$. Therefore, $q_\omega\prec q_0$. But we know that if $(a_i)$ and $(b_i)$ are families of mutually orthogonal projections with $a_i\prec b_i$ for each $i$, then $\sum_ia_i\prec\sum_ib_i$. % HERE why
		In particular, picking some bijection between $\omega+1$ and $\omega$, we have 
		\[p=\sum_{n\in(\omega+1)}q_n\prec\sum_{n\in\omega}q_n\le p\], and hence $\sum_{n\in\omega}q_n\approx p$. Similarly, $p\approx\sum_{n\in\omega}q_{2n}\approx\sum_{n\in\omega}q_{2n+1}$, so letting $q=\sum_{n\in\omega}q_{2n}$, we have $q\le p$ and $q,p-q\approx p$. 

		By repeating the above with $p_0=q$, we may as well assume that $p_0\approx p-p_0\approx p$ in the first place. Further extend $(q_n)_{n\in\omega}$ to a maximal family of mutually orthogonal projections $(r_n)_{n\in\omega}$ with each $r_n\prec p$; this family is still countable, again by countable decomposability of $M$. If $1-\sum_{n\in\omega}r_n\neq 0$, then just as before, we can find $r_\omega\le 1-\sum_{n\in\omega}r_n$ such that $r_\omega\prec p$, contradicting maximality. Therefore, $1=\sum_{n\in\omega}r_n$. Since each $p_n\approx p_0\approx p$, we have 
		\[1=\sum_{n\in\omega}r_n\prec\sum_{n\in\omega}p_n\le p\]
	 so we have $1\approx p$. 

  This means that there is a partial isometry $u$ with $u^*u=1$ and $uu^*=p$, so that $1=u^*(uu^*)u$ is conjugate to $p$ in $M$. If $I\subseteq M$ is a non-zero two-sided ideal, then by a previous application of the spectral theorem, % HERE was that why? Say more?
		there is a non-zero projection $p\in I$, and consequently, $1\in I$ and $I=M$.
		
		\end{proof}
 \item 
		A positive linear functional $\varphi \in M^*$ is called \textit{completely additive} if for any family of pairwise orthogonal projections $\left( p_i \right)$, $\varphi \left( \sum p_i \right) = \sum \varphi \left( p_i \right)$, where $\sum p_i$ converges SOT.\\
		Suppose $\varphi , \psi \in M^*$ are completely additive and $p \in P \left( M \right)$ such that $\varphi \left( p \right) < \psi \left( p \right)$.  Then there is a nonzero projection $q \leq p$ such that $\varphi \left( q x q \right) < \psi \left( q x q \right)$ for all $x \in M_+$.\\\
		\proof
		Let $\mathcal E = \left\{ e_i \mid i \in I \right\}$ be a maximal family of mutually orthogonal projections such that $e_i \leq p$ and $\psi \left( e_i \right) \leq \varphi \left( e_i \right)$ for all $i \in I$.\\
		Let $e = \bigvee_{i \in I} e_i$.  Then

		\begin{align*}
				\psi \left( e \right)%
				&= \psi \left( \bigvee_{i \in I} e_i \right)\\
				&= \psi \left( \sum_{i \in I} e_i \right), \quad \text{ since the $e_i$ are mutually orthogonal}\\
				&= \sum_{i \in I} \psi \left( e_i \right), \quad \text{ since $\psi$ is completely additive}\\
				&\leq \sum_{i \in I} \varphi \left( e_i \right)\\
				&= \varphi \left( \sum_{i \in I} e_i \right), \quad \text{ since $\varphi$ is completely additive}\\
				&= \varphi \left( \bigvee_{i \in I} e_i \right)\\
				&= \varphi \left( e \right),
		\end{align*}

		so that $\psi \left( e \right) \leq \varphi \left( e \right)$.  Note that this tells us that $e \neq p$.\\
		Take $q = p - e \neq 0$.  Then for all $r \leq q$, we must have $\psi \left( r \right) < \varphi \left( r \right)$, or else $\mathcal E \cup \left\{ r \right\}$ would be a larger family satisfying the stated properties, contradicting maximality of $\mathcal E$.\\
		Next, note that $\psi - \varphi \geq 0$, because on positive linear combinations of mutually orthogonal projections, we have

		\begin{align*}
				\left( \psi - \varphi \right) \left( \sum_{i = 1}^n \alpha_i p_i \right)%
				&= \sum_{i = 1}^n \alpha_i \left( \psi - \varphi \right) \left( p_i \right) \geq 0,
		\end{align*}

		since $\left( \psi - \varphi \right) \left( p_i \right) \geq 0$ for all $i = 1 , \ldots , n$.  Since positive linear combinations of mutually orthogonal projections are dense in $M_+$, it follows that $\psi - \varphi \in M_+$.\\
		Now take $x \in M_+$ such that $q x q \neq 0$.\\
		Then we can find $\alpha > 0$ such that $\spec \left( q x q \right) \cap \left[ \alpha , \infty \right) \neq \emptyset$.\\
		Then since $\psi - \varphi \geq 0$, we have

		\begin{align*}
				\left( \psi - \varphi \right) \left( x \right)%
				&\geq \left( \psi - \varphi \right) \left( \alpha 1_{\left[ \alpha , \infty \right)} \left( x \right) \right)\\
				&= \alpha \left( \psi - \varphi \right) \left( 1_{\left[ \alpha , \infty \right)} \left( x \right) \right)\\
				&> 0
		\end{align*}

		so that $\varphi \left( q x q \right) < \psi \left( q x q \right)$ for all $x \in M_+$ with $q x q \neq 0$.\\

	\item
		Show that the following conditions are equivalent for a positive linear functional $\varphi \in M^*$ for a von Neumann algebra $M$:

		\begin{enumerate}
		\item $\varphi$ is $\sigma$-WOT continuous,
		\item $\varphi$ is \textit{normal}: $x_\lambda \nearrow x$ implies $\varphi \left( x_\lambda \right) \nearrow \varphi \left( x \right)$, and
		\item $\varphi$ is completely additive.
		\end{enumerate}

		\proof
		Let $p \in P \left( M \right)$ be nonzero.\\
		Then we can find $\xi \notin \ker \left( p \right)$ such that $\varphi \left( p \right) < \left\langle p \xi , \xi \right\rangle$ ($p \xi \neq 0$, and we can scale).  Define $\psi = \left\langle \cdot \xi , \xi \right\rangle$.\\
		Then by problem 87, there exists a nonzero projection $q \leq p$ such that

		\begin{align*}
				\varphi \left( q x q \right) < \left\langle x q \xi , q \xi \right\rangle
		\end{align*}

		for all $x \geq 0$ with $q x q \neq 0$.\\
		Now we claim that $x \mapsto \varphi \left( x q \right)$ is $\sigma$-WOT continuous.  To see this, suppose $x_\lambda \to x$ $\sigma$-WOT, and hence $x_\lambda \to x$ WOT as well.  Then

		\begin{align*}
				\left| \varphi \left( x q \right) - \varphi \left( x_\lambda q \right) \right|^2%
				&= \left| \varphi \left( \left( x - x_\lambda \right) q \right) \right|^2\\
				&\leq \varphi \left( q \left( x - x_\lambda \right)^* \left( x - x_\lambda \right) q \right) \varphi \left( 1 \right)\\
				&\leq \left\langle \left( x - x0\lambda \right) q \xi , q \xi \right\rangle \varphi \left( 1 \right)\\
				&\to 0
		\end{align*}

		where the first inequality comes form Cauchy-Schwarz for the sesquilinear form $\left( x , y \right) \coloneq \varphi \left( y^* x \right)$, and the second inequality comes from the definition of $q$.  The convergence is exactly the definition of WOT convergence, so we conclude that $\varphi \left( \cdot q \right)$ is $\sigma$-WOT continuous.\\
		Now take $\left( q_i \right)_{i \in I}$ to be a maximal family of mutually orthogonal projections such that $x \mapsto \varphi \left( x q_i \right)$ is $\sigma$-WOT continuous.\\
		We claim that $\sum_{i \in I} q_i = 1$.  Take $p = 1 - \sum_{i \in I} q_i$, and suppose $p \neq 0$.  Then the construction above using this $p$ gives us a nonzero $q \leq p$ such that $\varphi \left( \cdot q \right)$ is $\sigma$-WOT continuous, which contradicts maximality.\\
		Define $\varphi_F \left( x \right) \coloneq \sum_{i \in F} \varphi \left( x q_i \right)$.  Then each $\varphi_F$ is $\sigma$-WOT continuous, and thus each $\varphi_F$ is $\sigma$-WOT continuous, so $\varphi_F \in M_*$.\\
		It now suffices to show that $\varphi_F \to \varphi$ in norm, so that $\varphi$ is also $\sigma$-WOT continuous, since $M_*$ is norm closed.\\
		Let $\epsilon > 0$.  Choose $F \subseteq I$ finite such that $\varphi \left( \sum_{i \notin F} q_i \right) < \frac{\epsilon^2}{\left\| \varphi \right\| + 1}$.  Then

		\begin{align*}
				\left| \varphi \left( x \right) - \varphi_F \left( x \right) \right|^2%
				&= \left| \varphi \left( x \right) - \sum_{i \in F} \varphi \left( x q_i \right) \right|^2\\
				&= \left| \varphi \left( \sum_{i \in I} x q_i - \sum_{i \in F} x q_i \right) \right|^2\\
				&= \left| \varphi \left( x \sum_{i \in F} q_i \right) \right|^2\\
				&\leq \varphi \left( x x^* \right) \varphi \left( \sum_{i \notin F} q_i \right)\\
				&\leq \left( \epsilon \left\| x \right\| \right)^2,
		\end{align*}

		and thus $\varphi_F \to \varphi$ in norm.\\
\end{enumerate}

\end{document}          
