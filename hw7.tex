\documentclass[a4paper,10pt]{report}
\usepackage[utf8]{inputenc}
\usepackage{amsmath}
\usepackage{amssymb}
\usepackage{amsthm}
\usepackage{mathtools}
\usepackage{fancyhdr}
\usepackage{enumitem}
\usepackage[top=1in,left=1in,right=1in]{geometry}
\usepackage{mathrsfs}
\usepackage{bm}

\setenumerate{listparindent=\parindent,topsep=\parskip}
% \setlist[enumerate]{topsep=\parskip}
\setlist[enumerate,2]{label=(\arabic*)}
\setlist[enumerate,3]{label=(\alph*)}

% \newcommand{\set}[1]{{\{#1\}}}
\newcommand{\ggen}[1]{\langle#1\rangle}
\newcommand{\pn}[2]{||#1||_{#2}}
\newcommand{\bpn}[2]{\left|\left|#1\right|\right|_{#2}}
\newcommand{\norm}[1]{||#1||}
\newcommand{\bnorm}[1]{\left|\left|#1\right|\right|}
\DeclarePairedDelimiter{\ceil}{\lceil}{\rceil}
\DeclarePairedDelimiter{\floor}{\lfloor}{\rfloor}
\DeclarePairedDelimiter{\set}{\{}{\}}
\DeclarePairedDelimiter{\abs}{|}{|}
\DeclarePairedDelimiter{\ket}{|}{\rangle}
\DeclarePairedDelimiter{\bra}{\langle}{|}

\newcommand{\op}[1]{\mathring{#1}}
\newcommand{\cl}[1]{\overline{#1}}
\newcommand{\ol}[1]{\overline{#1}}

\renewcommand{\mod}{\text{ mod }}

\renewcommand{\O}{\operatorname{O}} % Bound otherwise
\renewcommand{\o}{\operatorname{o}}
\newcommand{\T}{\text{yes}}
\newcommand{\F}{\text{no}}

\newcommand{\Z}{\mathbb{Z}}
\newcommand{\R}{\mathbb{R}}
\newcommand{\N}{\mathbb{N}}
\newcommand{\C}{\mathbb{C}}
\newcommand{\Q}{\mathbb{Q}}
\newcommand{\Prob}{\mathbb{P}}
\newcommand{\E}{\mathbb{E}}
\newcommand{\Ba}{\ensuremath Ba}
\newcommand{\del}{\partial}
\newcommand{\tens}{\otimes}
\newcommand{\rar}[2][]{\overset{#2}{\underset{#1}{\longrightarrow}}}

\DeclareMathOperator{\img}{img}
\DeclareMathOperator{\sgn}{sgn}
\DeclareMathOperator{\fop}{int}
\DeclareMathOperator{\fcl}{cl}
% \DeclareMathOperator{\lg}{lg}
\DeclareMathOperator{\vspan}{span}
\DeclareMathOperator{\diam}{diam}
\DeclareMathOperator{\id}{id}
% \DeclareMathOperator{\Im}{Im}
% \DeclareMathOperator{\Rp}{Rp}
\DeclareMathOperator{\rng}{rng}
\DeclareMathOperator{\Spec}{spec}
\DeclareMathOperator{\Rng}{Rng}
\DeclareMathOperator{\Cov}{Cov}
\DeclareMathOperator{\Var}{Var}
\DeclareMathOperator{\Bernoulli}{Bernoulli}
\DeclareMathOperator{\Normal}{Normal}
\DeclareMathOperator{\Uniform}{Uniform}
\DeclareMathOperator{\Binom}{Binomial}
\DeclareMathOperator{\mgf}{mgf}
\DeclareMathOperator{\Supp}{Supp}
\DeclareMathOperator{\Tr}{Tr}
\DeclareMathOperator{\tr}{tr}

\newcommand{\cat}[1]{(\bm{#1})}

\providecommand{\Alpha}{A}

\newtheorem*{lemma*}{Lemma}

% Roman numerals
\makeatletter
\newcommand{\Romnum}[1]{\expandafter\@slowromancap\romannumeral #1@}
\makeatother
\newcommand{\factor}[1]{\text{\Romnum{#1}}}

\begin{document}
% \maketitle

\pagestyle{fancy}	
\fancyhf{} % Reset headers and footers
\lhead{Ethan Ackelsberg, Zachery Dell, Peter Huston\\
Functional Analysis 2\\
\today}
\setlength{\headheight}{60pt}

\begin{center}
	\textbf{Homework 7}
\end{center}

\begin{enumerate}
		\setcounter{enumi}{83}
 \item 
 
 Let $S_{\infty}$ be the group of finite permutations of $\N$.
		\begin{enumerate}
			\item	Show that $S_{\infty}$ is ICC. Deduce that $LS_{\infty}$ is a $\factor{2}_1$ factor.
			\begin{proof}
			Fix $n\ge 2$. We may naturally identify $S_n\cong\set{\phi\in S_{\infty}:\phi(m)=m\text{ for all }m>n}\subseteq S_\infty$. There are infinitely many subgroups conjugate to $S_n$ in $S_\infty$ with trivial intersection: if $\phi_{n,k}=\prod_{j=1}^n(j,kn+j)$, then $\phi_{n,k}S_n\phi_{n,k}\bigcap\phi_{n,i}S_n\phi_{n,i}$ is trivial whenever $i\neq j$. Since every $\phi\in S_\infty$ lies in some $S_n$, if $\phi$ is not the identity, there are infinitely many distinct conjugates of $\phi$ in $S_\infty$. Therefore, $S_\infty$ is ICC. We proved in class that this makes $LS_\infty$ a type $\factor{2}_1$ factor. 
			\end{proof}
			\item	Give an explicit description of a projection with trace $k^{-n}$ for arbitrary $n, k \in \N$.
			\begin{proof}
			Let an integer $m=k^n$ be given. Let $H$ be a subgroup of $S_\infty$ of order $m$, for example generated by a cycle of length $m$. Set $p=m^{-1}\sum_{g\in H}\lambda_g$. Since $\lambda_{g^{-1}}=\lambda_g^*$, this $p$ is self-adjoint. Also, $p^2=m^{-2}\sum_{(g,h)\in H\times H}\lambda_g\lambda_h=m^{-1}\sum_{k\in H}\lambda_k=p$. Therefore, $p\in P(L\Gamma)$. Of course, $\tr(p)=\ggen{p\delta_e,\delta_e}=m^{-1}\sum_{g\in H}\ggen{\delta_g,\delta_e}=m^{-1}$, as desired. 
			\end{proof}
			\item	Find an increasing sequence $F_n \subseteq LS_{\infty}$ of finite dimensional von Neumann subalgebras such that $LS_{\infty} = \left( \bigcup_{n=1}^{\infty}{F_n} \right)$.
			\begin{proof}
			Since $S_\infty=\bigcup_{n\in\N}S_n$, we have $\C S_\infty=\bigcup_{n\in\N}\C S_n$, and $LS_\infty=(\C S_\infty)''$ by definition. 
			\end{proof}
		\end{enumerate}
		\setcounter{enumi}{85}
 \item 
 
 Let $M$ be a factor.
 Prove that if $M$ is finite or purely infinite, then $M$ is algebraically simple.
 \begin{proof}
 	We will deal with the three cases separately:
	type $\factor{1}_n$ for $n \in \N$, type $\factor{2}_1$, and type \factor{3}.
	First suppose $M$ is type $\factor{1}_n$.
	Then $M \cong M_n(\C)$.
	Suppose $I \subseteq M_n(\C)$ is a nonzero two-sided ideal.
	Let $A \in I$ be a nonzero matrix.
	Then taking $\left(e_i\right)_{i=1}^n$ as the standard basis vectors,
	we have $\alpha = \bra{e_i}A\ket{e_j} \ne 0$ for some $1 \le i, j \le n$.
	Now observe that
	$$p_k = \ket{e_k} \bra{e_k} = \frac{1}{\alpha}\ket{e_k}\bra{e_i} A \ket{e_j} \bra{e_k} \in I$$
	for $1 \le k \le n$.
	Summing over $k$, we have $1 = \sum_{k=1}^n{p_k} \in I$.
	
	Now suppose $M$ is type $\factor{2}_1$.
	Let $\tr$ be a $\sigma$-WOT continuous (faithful) tracial state on $M$.
	Let $I \subseteq M$ be a nonzero two-sided ideal.
	By the Borel functional calculus, $I$ contains a nonzero projection $p$.
	We claim that for any projection $q \in P(M)$, if $\tr(q) \le \tr(p)$, then $q \in I$.
	Let $q$ be such a projection.
	Then $q \preceq p$, so there is a partial isometry $u \in M$
	such that $uu^* = q$ and $u^*u \le p$.
	Since $I$ is a two-sided ideal, we have
	$$q = uu^* = u(u^*u)u^* = u(u^*up)u^* \in I.$$
	This proves the claim that $I \supseteq \{q \in P(M) : \tr(q) \le \tr(p)\}$.
	If $\tr(p) \ge \frac{1}{2}$, this implies immediately that $1 - p \in I$, whence $1 = p + (1 - p) \in I$.
	Suppose $\tr(p) < \frac{1}{2}$.
	Then $p \preceq 1 - p$, so there is a projection $q \le 1 - p$ with $q \approx p$.
	By the above claim, $q \in I$, so $p + q \in I$ and $\tr(p + q) = 2\tr(p)$.
	Thus, we may find a projection in $I$ with trace double that of $p$.
	Repeating this if necessary, we eventually arrive at a projection of trace at least $\frac{1}{2}$,
	so by the previous case, $I = M$.
 \end{proof}
 
 \item 
 \item 
\end{enumerate}

\end{document}          
